\chapter{Differential graded categories}
\begin{flushright}
  Pieter Belmans
\end{flushright}

\section{Introduction}
The notion of triangulated category lies at the heart of homological algebra. This type of category is essential in the study of derived categories and stable homotopy categories of spectra. But there are some problems with this concept: the cone construction is not-functorial. Already in Verdier's PhD thesis there is the result that if the cones in a (countably) (co)complete triangulated category are functorial that category is necessarily semisimple abelian \cite[proposition II.1.2.13]{verdierphd}. As not every triangulated category is abelian \cite[exercise 1.4.5]{weibel} this is a problem\expandthis{why exactly is this a problem?}. To put it more bluntly:
\begin{quote}
  ``This `nonfunctoriality of a cone' is the first symptom that something is going wrong in the axioms of a triangulated category.''
\end{quote}
\begin{flushright}
  \cite[section IV.7]{gelfandmanin}
\end{flushright}
Another symptom is that certain theorems involving derived categories are ``weak'', in the sense that one might suspect them to hold in a more general context. By enriching the category before applying the constructions from homological algebra, we can retain enough information about the original category to solve some of the issues.

In a more abstract context: up to now we have seen that there are different models for~$(\infty,1)$\dash categories. The models we have discussed so far are quasicategories, relative categories using simplicial localisation, Segal categories and complete Segal spaces. These are all models for the general theory of~$(\infty,1)$\dash categories. One could restrict himself on the other hand to certain subclasses of~$(\infty,1)$\dash categories. An important example of this phenomenon are model categories, which provide a model for certain~$(\infty,1)$\dash categories that are both complete and cocomplete. Whether they can model \emph{all} complete and cocomplete~$(\infty,1)$\dash categories is not known. Another possible subclass of~$(\infty,1)$\dash categories is the one modelled using differential graded categories. They are so called ``linear'' models, in the same sense that homological algebra is a linear version of homotopical algebra.

Whereas topological or simplicial categories are categories enriched over topological spaces or simplicial sets, a differential graded category is a category enriched over (co)chain complexes of~$k$\dash modules, for~$k$ a commutative ring. So we restrict ourselves from~$\infty$\dash groupoids to so called abelian and fully strict~$\infty$\dash groupoids, these are exactly the~$\infty$\dash groupoids modelled by chain complexes.

The goal of this expos\'e is to
\begin{enumerate}
  \item introduce dg~categories;
  \item discuss the model category structures on the category of dg~categories;
  \item give some applications of this machinery;
  \item explain how to construct a dg~category from a~$(\infty,1)$\dash category.
\end{enumerate}

The main expository references for dg~categories are \cite{lnm2008,keller}. The results in this expos\'e are obtained in \cite{toen}. Other interesting papers are \cite{toen-vaquie}\expandthis.


\section{Differential graded categories}
From now on we fix a commutative ring~$k$. Every construction is relative to this base ring. Whenever the ring is required to be a field it will be specified. We moreover use cochain complexes, i.e.\ the degree of the differential is~$1$, but the exposition can be done completely the same using chain complexes and morphisms of degree~$-1$.

As already suggested, the definition of a dg~category is easy. We will freely use the language of enriched categories, but whenever a down-to-earth interpretation can be made it will be given.
\begin{definition}
  \label{definition:dg-category-enriched}
  A \emph{dg~category} is a category enriched over cochain complexes of~$k$\dash modules.
\end{definition}
A cochain complex can equivalently be considered as a dg~$k$\dash module: it is a graded~$k$\dash module equipped with a differential~$\dd$ of degree~$1$. The monoidal structure of~$k\textrm{-}\Mod$ implies that the composition in a dg~category~$\mathcal{C}$
\begin{equation}
  \Hom_{\mathcal{C}}(Y,Z)^\bullet\otimes_k\Hom_{\mathcal{C}}(X,Y)^\bullet\to\Hom_{\mathcal{C}}(X,Z)^\bullet
\end{equation}
is a morphism of dg~$k$\dash modules. Remark that the tensor product of graded~$k$\dash modules~$V^\bullet$ and~$W^\bullet$ is given by
\begin{equation}
  (V^\bullet\otimes_k W^\bullet)^n\coloneqq\bigoplus_{\mathclap{p+q=n}}V^p\otimes_k W^q.
\end{equation}

\expandthis{discuss dg modules}

\section{Model category structures on \texorpdfstring{$\dgCat_k$}{dgCat\textunderscore k}}
It is possible to put at least two different model category structures on~$\dgCat_k$: one with quasi-equivalences as weak equivalences, and the other with Morita morphisms as weak equivalences.

We start with the ``canonical'' model category structure, where the quasi-equivalences are inverted.
\begin{definition}
  Let~$f\colon\mathcal{C}\to\mathcal{D}$ be a morphism in~$\dgCat_k$, i.e.\ a dg~functor between dg~categories. It is said to be \emph{quasi-fully faithful} if for all~$X,Y\in\Ob(\mathcal{C})$ the map
  \begin{equation}
    f_{X,Y}\colon\Hom_{\mathcal{C}}(X,Y)^\bullet\to\Hom_{\mathcal{D}}(f(X),f(Y))^\bullet
  \end{equation}
  of cochain complexes is a quasi-isomorphism. It is said to be \emph{quasi-essentially surjective} if the induced functor
  \begin{equation}
    \HH^0(f)\colon\HH^0(\mathcal{C})\to\HH^0(\mathcal{D})
  \end{equation}
  on the level of ($k$\dash linear) categories is essentially surjective.
\end{definition}
These are versions of the two conditions necessary to define the equivalence of categories that are compatible with the enrichment, hence we can define the following.
\begin{definition}
  Let~$f\colon\mathcal{C}\to\mathcal{D}$ be a morphism in~$\dgCat_k$. It is said to be a \emph{quasi-equivalence} if it is both quasi-fully faithful and quasi-essentially surjective.
\end{definition}
We now introduce the fibrations in this model category structure.
\begin{definition}
  \label{definition:quasi-fibration}
  Let~$f\colon\mathcal{C}\to\mathcal{D}$ be a morphism in~$\dgCat_k$. It is said to be a \emph{quasi-fibration}\index{quasi-!fibration} if
  \begin{enumerate}
    \item for all~$X,Y\in\Ob(\mathcal{C})$ the map
      \begin{equation}
        f_{X,Y}\colon\Hom_{\mathcal{C}}(X,Y)^\bullet\to\Hom_{\mathcal{D}}(f(X),f(Y))^\bullet
      \end{equation}
      is a fibration in~$\Ch(k\mhyphen\Mod)$ for the projective model category structure on~$\Ch(k\mhyphen\Mod)$, i.e.\ is it an epimorphism in every degree;
    \item for all~$X\in\Ob(\mathcal{C})=\Ob(\HH^0(\mathcal{C}))$ and for all isomorphisms~$v\colon\HH^0(f)(X)\to Y$ in~$\HH^0(\mathcal{D})$ we can lift~$v$ to an isomorphism~$u\colon X\to Y$ in~$\HH^0(\mathcal{C})$ such that~$\HH^0(f)(u)=v$.
  \end{enumerate}
\end{definition}

\begin{theorem}[\cite{tabuada}]
  \label{theorem:quasi-equivalences-model-category-structure}
  If we take take the quasi-equivalences as weak equivalences, the quasi-fibrations as fibrations, and the dg~functors satisfying the right lifting property with respect to the trivial fibrations as cofibrations we obtain a model category structure on~$\dgCat_k$.
\end{theorem}

In the second model category structure we use a bigger class of weak equivalences, the same fibrations and therefore a smaller class of cofibrations. Its main uses are \expandthis{what is it good for?}
\begin{definition}
  Let~$\mathcal{C}$ be a dg~category. Its \emph{derived category of dg~$\mathcal{C}$\dash modules}~$\mathbf{D}\dgMod{\mathcal{C}}_k$ is the localisation of~$\dgMod{\mathcal{C}}_k$ with respect to the quasi-isomorphisms.
\end{definition}
Associated to a dg~functor~$f\colon\mathcal{C}\to\mathcal{D}$ is the restriction functor
\begin{equation}
  f^*\colon\dgMod{\mathcal{D}}_k\to\dgMod{\mathcal{C}}_k
\end{equation}
given by composition. This yields the following definition.
\begin{definition}
  Let~$f\colon\mathcal{C}\to\mathcal{D}$ be a morphism in~$\dgCat_k$. It is said to be a \emph{Morita morphism} if
  \begin{equation}
    \mathbf{D}f^*\colon\mathbf{D}(\dgMod{\mathcal{D}}_k)\to\mathbf{D}(\dgMod{\mathcal{C}}_k)
  \end{equation}
  is an equivalence of categories.
\end{definition}

\begin{theorem}[\cite{tabuada-2}]
  \label{theorem:morita-morphisms-model-category-structure}
  If we take take the Morita morphisms as weak equivalences, the quasi-fibrations as fibrations, and the dg~functors satisfying the right lifting property with respect to the trivial fibrations as cofibrations we obtain a model category structure on~$\dgCat_k$.
\end{theorem}

\begin{remark}
  From this point on, whenever we denote~$\Ho(\dgCat_k)$ it will be the homotopy category of~$\dgCat_k$ with respect to the model category structure using quasi-equivalences. This is in line with \cite{toen} on which most of the results in this expos\'e are based.
\end{remark}

\section{Mapping spaces}
The following section is a discussion of the main technical result in \cite{toen}. Its actual proof is rather long and technical, and there is not enough room here to discuss even the necessary definitions and notations. One needs cosimplicial resolution functors, restrictions to specific subcategories, diagonals of bisimplicial sets and nerves just to write down the statement. So immediately its interpretation in terms of the homotopy category is given.\expandthis{rqr not defined yet}

\begin{theorem}
  \label{theorem:fundamental-bijection}
  Let~$\mathcal{C}$ and~$\mathcal{D}$ be small dg~categories. We have a functorial bijection
  \begin{equation}
    \label{equation:fundamental-bijection}
    \Hom_{\Ho(\dgCat_k)}(\mathcal{C},\mathcal{D})\cong\Isom\left( \Ho\left( \dgMod{(\mathcal{C}\LLLotimes\mathcal{D}^\opp)}_k^\rqr \right) \right).
  \end{equation}
\end{theorem}
Hence every dg~functor in the homotopy category can be represented in a unique way by a right quasi-representable bimodule. If one is familiar with Morita theory this might ring a bell. The remainder of \cite{toen} is based on the technical result that underlies this theorem. One of these is the 

\section{Monoidal structure}
Recall that there exists a closed monoidal structure on~$\dgCat_k$. Unfortunately, this structure does not descend to the homotopy category, as the tensor product of two cofibrant dg~categories is not necessarily cofibrant. So in order to prove the existence of a closed monoidal structure one needs to do some work.\expandthis{Int not yet defined}
\begin{theorem}
  \label{theorem:internal-Hom}
  We can equip the monoidal category~$(\Ho(\dgCat_k,-\LLLotimes-))$ with an internal Hom-object which we'll denote~$\RRRHHom$, hence it is a closed monoidal category. This internal Hom is moreover characterised by
  \begin{equation}
    \RRRHHom(\mathcal{C},\mathcal{D})\cong\Int\left( \dgMod{(\mathcal{C}\LLLotimes\mathcal{D}^\opp)}_k^\rqr \right)
  \end{equation}
  for~$\mathcal{C},\mathcal{D}\in\Ob(\Ho(\dgCat_k))$.
\end{theorem}
So we have the derived tensor-Hom adjunction
\begin{equation}
  \label{equation:tensor-Hom-adjunction}
  \Hom_{\Ho(\dgCat_k)}(\mathcal{C}\LLLotimes\mathcal{D},\mathcal{E})\cong\Hom_{\Ho(\dgCat_k)}(\mathcal{C},\RRRHHom(\mathcal{D},\mathcal{E}))
\end{equation}
for all~$\mathcal{C},\mathcal{D}$ and~$\mathcal{E}$ dg~categories.

This closed monoidal structure is moreover compatible with the derived simplicial structure as introduced before.
\begin{corollary}
  \label{corollary:simplicial-compatibility}
  Let~$\mathcal{C}$ and~$\mathcal{D}$ be two dg~categories. Let~$K$ be a simplicial set. Then we have the functorial isomorphism
  \begin{equation}
    K\LLLotimes_\simp(\mathcal{C}\LLLotimes\mathcal{D})\cong(K\LLLotimes_\simp\mathcal{C})\LLLotimes\mathcal{D}
  \end{equation}
  in~$\Ho(\dgCat_k)$.
\end{corollary}
This corollary yields the following enriched tensor-Hom adjunction.
\begin{corollary}
  \label{corollary:internal-Hom-Map-adjointness}
  Let~$\mathcal{C},\mathcal{D},\mathcal{E}$ be dg~categories. Then we have the functorial isomorphism
  \begin{equation}
    \label{equation:internal-Hom-Map-adjointness}
    \Map_{\Ho(\dgCat_k)}^\ell(\mathcal{C}\LLLotimes\mathcal{D},\mathcal{E})\cong\Map_{\Ho(\dgCat_k)}^\ell(\mathcal{C},\RRRHHom(\mathcal{D},\mathcal{E}))
  \end{equation}
  in~$\Ho(\sSet)$.
\end{corollary}

\section{Applications}

\section{\texorpdfstring{$(\infty,1)$-}{(oo,1)-} and dg categories}
