\chapter{Derived loop spaces and de Rham cohomology}
\chapterprecistoc{\textup{by} Pietro Vertechi}
\begin{flushright}
  Pietro Vertechi
\end{flushright}

\begin{refsection}

\section{Topological spaces and groups in derived algebraic geometry}
\label{topology}
In this section we briefly discuss the concept of affinisation in
derived algebraic geometry as it will play a rather important role in studying the loop space of a stack. We work over the site of graded commutative dg algebras
concentrated in nonpositive degrees (which we denote $\dalg$) with the étale topology. We work
over a fixed base field k of characteristic 0 which is fixed for all time and often implicit. Whenever we say stack, we actually mean derived stack.
\subsection{Affinisation}

We want to introduce the notion of affine stack (Toën), which is a homotopical generalisation of the notion of affine scheme and will be useful for dealing
with CW complexes considered as derived stacks.
\begin{definition}
The category of affine stacks is the opposite of the category of dg algebras, without any coconnectivity assumptions: $Aff:=dgAlg^{op}$
\end{definition}

\begin{remark}
There is an evident map $\spec: Aff \to St $ given by the Yoneda functor: $\spec(R)(A):=\Hom_{cdga}(R,A)$
\end{remark}

As in the classical case, this functor has a left adjoint $\O: St \to Aff$ which is the only functor sending each derived affine scheme to the corresponding coconnective
dg algebra and preserving colimits.
The monad $\Aff:= \spec\circ\O$ is called the affinisation functor. Each stack X comes equipped with a universal arrow $X\to\Aff(X)$ through which every arrow
from X to an affine stack factorises.\\

Now we will try to understand what the affinisation functor does on CW complexes. In order to do that, we need to be able to understand how to interprete
CW complexes as derived stacks.
\begin{proposition}
There is an adjunction $sSets \rightleftarrows PrSt$ where the left adjoint sends each simplicial set 
to the corresponding constant prestack and the right adjoint sends each prestack X to the simplicial set X(k)
\end{proposition}

\begin{proof}
$\Map(S,X(k))=\Map(S,\Hom(\spec(k), X))=\Hom(\spec(k)\otimes S, X)$ and $\spec(k)\otimes S$ is the constant simplicial prestack associated to S, as colimits are taken objectwise
and the prestack $\spec(k)$ associates to each dg algebra a point.
\end{proof}

We can compose $sSets \rightleftarrows PrSt$ with the adjunction: $PrSt\rightleftarrows St$ between the forget and the sheafification functor.\\
We get maps $sSets \rightleftarrows St$ which allows us to consider simplicial sets as a particular case of derived stacks.
In particular, as left adjoints preserve colimits, and finite simplicial
sets can be expressed as finite colimits of copies of the point, we can recover finite simplicial sets (or equivalently finite CW complexes) as those stacks 
which are finite colimits of copies
of $\spec(k)$.\\
For example the circle $S^1$ is the derived stack:
$$\spec(k)\coprod_{\spec(k)\coprod \spec(k)}\spec(k)$$
Instead general simplicial sets correspond to arbitrary colimits of copies of $\spec(k)$.\\

\subsection{Cohomology}
In classical algebraic topology cohomology is represented by the Eilenberg-MacLane spaces $K(G,n)=B^nG$. We will be interested in cohomology with coefficient
in k, that is to say $G=\G_a$ is the additive group. What does $B^n\G_a$ look like? 

\begin{proposition}
$B^n\G_a\simeq \spec(\Sym k[-n])$
\end{proposition}
\begin{proof}
Let us consider the functor $\O$ as a contravariant functor from $St$ to $dgMod$. It sends pushout squares to pullback squares (which are also pushout squares
as the category $dgMod$ is stable).\\
If we consider the pushout diagram:

\begin{displaymath}
\xymatrix{
S^{d-1} \ar[r] \ar[d] &  pt \ar[d]\\
 pt \ar[r] & S^d\\
}
\end{displaymath}

It maps to the pushout/pullback diagram:

\begin{displaymath}
\xymatrix{
\O(S^{d}) \ar[r] \ar[d] &  k \ar[d]\\
 k \ar[r] & \O(S^{d-1})\\
}
\end{displaymath}

which allows us to calculate $\O(S^d)$ (as a dg module) recursively: $\O(S^d)=k\oplus k[-d]$. Therefore, for any $A \in\dalg$
$$\Hom(S^d\times \spec(A),\spec \Sym k[-n])\simeq \Hom_{dgAlg}(\Sym k[-n],\O(S^d)\otimes A])\simeq \Hom_{dgMod}(k[-n],A\oplus A[-d])$$
So it is clear that, if one points $\spec\Sym k[-n]$ with the zero map $\Sym k[-n]\to k$, one has:
$\Omega^d \spec \Sym k[-n]\simeq \spec\Sym k[-n+d]$ so $\pi_d(\spec \Sym k[-n])$ is trivial for $d<n$ and $\Omega^n(\spec \Sym k[-n])\simeq \G_a$.
As the first n-1 homotopy groups are trivial $\spec \Sym k[-n]\simeq B^n\Omega ^n\Sym k[-n]\simeq B^n\G_a$
%\begin{lemma}
%Let X be a stack, with only one nontrivial homotopy group, $\pi_n(X)$. Then $X\simeq B^n\pi_n(X)$.
%\end{lemma}
%\begin{proof}
%As X is n-1 connected, $X\simeq B^n\Omega^n X$. Furthermore $\pi_n(X)\simeq \pi_0(\Omega^n X)$ and it is also clear that the obvious map
%$\Omega^n X\to \pi_0(\Omega^n X)$ is an equivalence, as it induces isomorphisms on all the homotopy groups: on the 0-th it is obvious and the others are trivial.
%\end{proof}
%Applying the lemma to $X=\spec \Sym k[-n]$ we get the thesis.
\end{proof}


\begin{theorem}
For X a CW complex, $\O(X)\simeq C^*(X)$ where the right member of the equality is the cochain complex.
\end{theorem}

\begin{proof}
$H^n(X)=\Map(X,B^n\G_a)=\Hom_{dgAlg}(\Sym k[-n],\O(X))=H^n(\O(X))$. The natural map $C^*(X)\simeq C^*(\col\spec(k))\to \lim C^*(\spec(k))\simeq \lim k\simeq \O(X)$,
induces those isomorphisms on cohomology.\\
\end{proof}

\begin{remark}
Actually we have a particularly nice description of the algebra $\Sym k[-1]$: it is generated by one element of degree one $\epsilon$ and the algebra structure
is trivial as $\epsilon\epsilon=-\epsilon\epsilon$ (by graded-commutativity) so both sides are equal to 0.
So $\Sym k[-1]\simeq k[\epsilon]/(\epsilon^2)$
\end{remark}

\begin{remark}
So far we've seen that $B^n\G_a\simeq \spec \Sym k[-n]$ as stacks. As $\G_a$ is an abelian monoid, $B^n\G_a$ also has a monoid structure which induces the sum structure in
the n-th cohomology group. There is a comonoid structure on $\Sym k[-n]$ corresponding to the group structure on the $H^n$ for dg-algebras: the comultiplication is given by $\e\mapsto 1\otimes\e+\e\otimes 1$
and the counit by $\e\mapsto 0$. As $\spec$ is a monoidal functor (it sends colimits to limits, so in particular the tensor product is sent to the
direct product), $\spec \Sym k[-n]$ is also a monoid. I and it is tautological that this monoid structure induces the sum in the n-th cohomology group.\\
It is legitimate to ask whether $B^n\G_a$ and $\spec\Sym k[-n]$ are isomorphic as monoids. Fortunately, to prove that it is possible to use the same reasoning as above on
$B\spec\Sym k[-n]$: its first n homotopy groups are trivial and $\Omega^{n+1}B\spec\Sym k[-n]\simeq \G_a$, so $B\spec\Sym k[-n]\simeq B^{n+1}\G_a$. By
applying the $\Omega$ functor, one gets a monoid equivalence between $B^n\G_a$ and $\spec\Sym k[-n]$
\end{remark}

%\subsection{Monoids and monoid action}



%\begin{proposition}
%QC(X/G) is the category of G-equivariant quasi-coherent sheaves over X.
%\end{proposition}
 
%\begin{proof}
%We only need to prove it for an affine $X=\spec(A)$. In that case X/G is the colimit of a BG(A) shaped diagram where the only object is $\spec(A)$ and to each arrow we assign
%the corresponging element of G. As QC sends colimits to limits, the thesis follows.
%\end{proof}

%\begin{corollary}
%\label{classifier}
%In particular, the category of modules with G-action is simply $QC(*/G)=QC(BG)$
%\end{corollary}

\subsection{Actions of CW complexes}
There is of course a two-coloured operad which axiomatises an object and a monoid acting over it. Using the formalism developed in section \ref{weak structures}, we can
easily define the $\infty$-category of monoid stacks acting on a stack. If we take the sull subcategory of those functors for which the first object goes to a given
monoid G, we obtain the category of G-modules. 
We shall now consider the case of finite CW-complexes which are monoids, show that their affinisation is also a monoid in a canonical way and that the categories
of G-modules and Aff(G)-modules are canonically equivalent.

\begin{lemma}
Let G be a finite CW complexes and X a stack, then $\O(G\times X)\simeq \O(G)\otimes \O(X)$
\end{lemma}

\begin{proof}
Let us consider the two functors: $\O(-\times X)$ and $\O(-)\otimes \O(X)$. The first sends colimits to limits, so in particular pushout
squares to pullback squares. If we apply $\O(-)$ to a pushout square we get a pullback square which is also pushout as $Mod$ is stable. If then we 
apply $-\otimes \O(X)$ we get a pushout square which is also pullback.\\
The two functors are equal on $\spec(k)$ and, since finite CW complexes can be obtained starting from $\spec(k)$ with iterated pushouts, the two 
functors agree on finite CW complexes.
\end{proof}

\begin{corollary}
A monoid structure over a CW complex G induces canonically a comonoid structure on $\O(G)$ and a monoid structure over $\Aff(G)$. Furthermore
whenever we are given a G action on a stack X, we get a $\O(G)$ coaction on $\O(X)$ and an $\Aff(G)$ actions on $\Aff(X)$.\\
In particual, $G$ actions on affine derived schemes are the same as $\O(G)$ coaction on derived algebras or $\Aff(G)$ actions on affine derived schemes.
\end{corollary}

\begin{example}
\label{formality}
$G=S^1=B\mathbb{Z}$. There is an obvious group morphism $\mathbb{Z}\to \G_a$ sending $1\in\mathbb{Z}$ to the multiplicative unit of k. Such a morphism induces
a monoid morphism $S^1\simeq B\mathbb{Z}\to B\G_a\simeq \spec(\cerchio)$. By adjunction, we obtain a map $\O(S^1)\to \cerchio$
It is easy to verify that such map is a quasi-isomorphism. Moreover, as the adjunction (restricted to finite CW complexes) is monoidal and such map comes from a monoid
morphism, it is also a comonoid morphism. Concretely, this means that:
\begin{itemize}
 \item The dg Hopf algebra $\cerchio$ is a strict model for the comonoid $\O(S^1)$
\item $\Aff(S^1)$ is isomorphic to $B\G_a$ as monoid stacks.
\end{itemize}

\end{example}



\section{Loop space and the algebra of differential forms}
\label{loop}
\begin{definition}
The loop space $\mathcal{L}X$ of a derived stack X is the mapping stack $\Map(S^1,X)$
\end{definition}

\begin{remark}
If X is affine, then all map $S^1\to X$ factors through $S^1\to Aff(S^1)\simeq BG_a$, so\\ $\mathcal{L}X\simeq \Map(BG_a,X)$
\end{remark}


\begin{remark}
As $\displaystyle S^1\simeq *\coprod_{*\coprod *}*$, it follows immediately that $\mathcal{L}X\simeq X\times_{X\times X}X$.\\
Similarly, if $X=\spec(R)$ is an affine stack, $\mathcal{L}X\simeq \spec(R\otimes_{R\otimes R}R)$
\end{remark}

\subsection{Hochschild homology and the multiplicative HKR isomorphims}

For A a derived algebra, $A\otimes_{A\otimes A}A$ corresponds to what is classically known as its Hochschild chain complex. In this setting it is much easier to
prove the HKR isomorphism: $$A\otimes_{A\otimes A}A\simeq \Omega_A^{-\bullet}$$ where $\Omega_A^{-\bullet}=\Sym^{\bullet} \Omega^1_A[1]$ is the algebra of
differential forms.\\

To understand this isomorphism, it's important to consider the equivalence of categories $$S^1-Aff^{op}\simeq \cerchio-\dalg$$ 
given by the $\spec$, $\O$ adjunction (and by the formality of cochains of $S^1$, see example \ref{formality}).

There are the obvious functors: $\O :S^1-Aff^{op}\to \dalg$ and $forget: \O(S^1)-\dalg\to\dalg$, which commute with the $\spec$, $\O$ adjunction.
Moreover both functors have a left adjoint, giving the following commutative diagram:
\begin{displaymath}
\xymatrix{
S^1-Aff^{op} \ar@<2pt>[dr]  \ar@<2pt>[rr]^-\simeq&  & \cerchio-\dalg\ar@<2pt>[dl] \ar@<2pt>[ll]\\
& \dalg \ar@<2pt>[ur] \ar@<2pt>[ul]&
}
\end{displaymath}

It is almost tautological that the first left adjoint is the loop space functor $A\mapsto \L\spec A\simeq \spec(A\otimes_{A\otimes A}A)$.
Indeed, given a map $A\to O(X)$ one gets a map $X\to \spec(A)$ and, by consequence, a map $\L X\to \L\spec(A)$ and, if we precompose it by the action map $X\to \L X$
we get the desired morphism $X\to \L\spec(A)$. If instead we are given a map $X\to \L\spec(A)$ we must simply compose with map $\L\spec(A)\to \spec(A)$ sending a loop
to its value in the identity to get a map $X\to \spec(A)$ and so $A\to O(X)$.\\

The adjoint of the second forget functor is more interesting: it sends a derived algebra A to its algebra of differential forms. To see why it is so, we need the following:
\begin{lemma}
If R is a derived algebra together with a square-zero derivation $d:R\to R[-1]$ then the map $R\to R\oplus R[-1]$ which is the identity on the first component
and d on the second is a $\cerchio$ coaction map on R.
\end{lemma}

What the lemma is saying is that derived algebras with a square-zero derivation $d:R\to R[-1]$ (the so-called $\e$ dg algebras) correspond to algebras with a
strict $\cerchio$ coaction, so in particular $\Omega_A^{-\bullet}$ is a $\cerchio$-comodule in a natural way.

\begin{proposition}
$\L\spec A \simeq \spec\Omega_A^{-\bullet}$. $\spec\Omega_A^{-\bullet}$ is also called the odd tangent complex of A and denoted $\mathbb{T}_A[-1]$.
\end{proposition}

\begin{proof} $\Omega_A^{-\bullet}$ is a $\cerchio$-comodule, so $\spec\Omega_A^{-\bullet}$ is a $S^1$-module, so by the universal property we have a map of $S^1$-stacks
$\spec\Omega_A^{-\bullet}\to \L\spec A$. We want to check that it is an equivalence.\\

Let's consider any S point of $\spec A$: $x:\spec S\to \spec A$ and write the universal property of the cotangent complex:

\begin{displaymath}
\xymatrix{
\Hom_S(x^*\Omega_A, M) \ar[r] \ar[d] &  \Hom(A,S\oplus M) \ar[d]\\
 pt \ar[r] & \Hom(A,S)\\
}
\end{displaymath}

where M is any S-module. Indeed, it can be proved \cite{Lu} that the universal property of the cotangent complex holds without any connectivity assumptions on M.\\
Let's apply this universal property to the case $M=S[-1]$:

\begin{displaymath}
\xymatrix{
\Hom_S(x^*\Omega_A, S[-1]) \ar[r] \ar[d] &  \Hom(A,S\oplus S[-1]) \ar[d]\\
 pt \ar[r] & \Hom(A,S)\\
}
\end{displaymath}

The key point is that $\L \spec(A)(S)=\Map(B\G_a,\spec A)(S)=\Hom(\spec(\cerchio\otimes S),\spec A)=\Hom(A,S\oplus S[-1] )$.\\
Furthermore $\Hom_S(x^*\Omega_A, S[-1])\simeq \Hom_S(x^*\Omega_A[1], S)\simeq \Hom_{S-alg}(x^*\Omega_A^{-\bullet}, S)=x^*\mathbb{T}_X[-1](S)$\\
So the diagram we have found is actually:

\begin{displaymath}
\xymatrix{
x^*\mathbb{T}_A[-1](S) \ar[r] \ar[d] &  \L\spec(A)(S) \ar[d]\\
 pt \ar[r] & \spec(A)(S)\\
}
\end{displaymath}

and so the map we constructed is an isomorphism.
\end{proof}

\begin{corollary}
The left adjoint of the forgetful functor from $\cerchio-\dalg$ to $\dalg$ is the algebra of differential forms.
\end{corollary}

\begin{remark}
This is to be read as a strictification result, because if we started with algebra with a strict $\cerchio$-coaction (that is to say $\e$ dg algebras) the thesis would be
obvious.
\end{remark}

So, to summarise our results:

\begin{theorem} Given A a derived algebra, we have:
\begin{itemize}
 \item $A\otimes_{A\otimes A}A\simeq \Omega_A^{-\bullet}$ as derived algebras.
 \item The map $Id\oplus d:\Omega_A^{-\bullet}\to\Omega_A^{-\bullet}\oplus \Omega_A^{-\bullet}[-1]$ is a
 $\cerchio$-coaction.
\item Applying the $\spec$ functor to everything we get the map $$S^1\times \L\spec(A)\to B\G_a\times \L \spec(A)\to\L\spec(A)$$ which is isomorphic to
the rotation action of $S^1$ on $\L \spec(A)$.
\end{itemize}
\end{theorem}

Actually, what we've done can be generalised, with very little effort, to derived schemes, thanks to the following:

\begin{proposition} \cite{Be-Na}
For X a derived scheme and $U\to X$ a Zariski open, the induced map $\L U\to \L X$
is also Zariski open. Furthermore the square:
\begin{displaymath}
\xymatrix{
\L U \ar[r] \ar[d] &  \L X \ar[d]\\
 U \ar[r] & X\\
}
\end{displaymath}
is a pullback
\end{proposition}

Therefore the isomorphism $\L X\simeq \mathbb{T}_X[-1]$ holds, more generally, for derived schemes.

\begin{remark}
It is important to notice that in the HKR isomorphism, we do not consider the de Rham differential as a differential: the differential on $\Omega_A^{-\bullet}$ is the one
given by the differential of the cotangent complex. Fortunately the information of the differential is not lost: it is encoded in the $S^1$ action on $\L X$, so to
find for example negative cyclic homology (see \cite{connes} or \cite{shiftedsymplectic}) one simply takes $S^1$ homotopy invariants functions on $\L X$ that is to say
$\O_{\L X}^{S^1}$
\end{remark}

\nocite{to3, ToVedeRham}
\printbibliography[heading = local]

\end{refsection}
