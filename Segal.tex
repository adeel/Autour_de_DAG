\chapter{Segal Spaces and Segal Categories}

\begin{refsection}

In this talk, we will discuss two models for $(\infty,1)$-categories in detail, namely complete Segal spaces and Segal categories. We will focus on the homotopy theory of these two models, as well as briefly sketching their relations to other models.

\begin{flushright}
Yan Zhao
\end{flushright}

\section{Preliminaries}
\subsection{Simplicial space}
In this paper, a {\bf space} will always refer to a simplicial set. Let $\mathcal S$ be the category of spaces, which we endow with the standard model category structure, i.e. a weak equivalence is a homotopy weak equivalence, a cofibration is an injection and a fibration is a Kan fibration. Let $\Delta^n$, $\partial\Delta^n$ and $\Lambda^n_k$ be the standard $n$-simplex, the boundary of the standard $n$-simplex and the $k$-th horn of the $n$-simplex (boundary of the $n$-simplex with the $k$-th face removed) respectively. For any $X,Y\in\cal S$, the mapping space is the function complex $\Map_\mathcal S(X,Y)$, where the $n$-simplex can be given by the set of all maps $X\times\Delta^n\to Y$.

Let $\Delta\subset\Cat$ be the category of categories consisting of objects $[n]$, with the usual face and degeneracy maps $d^i$ and $s^i$. A {\bf simplicial space} is thus a functor $X:\Delta^{op}\to\mathcal S$, which sends $[n]\mapsto X_n$, with face and degeneracy maps $d_i:X_n\to X_{n-1}$ and $s_i:X_n\to X_{n+1}$. Let $s\mathcal S$ be the category of simplicial spaces.

Note that a simplicial space can also be seen as a bisimplicial set $X:\Delta^{op}\times\Delta^{op}\to \Set$ with two sets of arrows $d_i$ and $s_i$. Let ${\mathcal S}^{(2)}$ denote the category of bisimplicial sets. It is clear that there is an isomorphism of categories between $s\mathcal S$ and ${\mathcal S}^{(2)}$. The former notation is more convenient for the discussion of homotopy theory while the latter in the comparison theorems between different models of $\infty$-categories. We will freely interchange between the two characterisations.

We can identify $\mathcal S$ as a full subcategory of $s\mathcal S$ by sending each simplicial set $K$ to the constant simplicial space $[n]\mapsto K$ with $d_i$ and $s_i$ being the identity maps. The category $s\mathcal S$ can be enriched over simplicial sets in a compatible way with the enrichment of $\mathcal S$. For any $X,Y\in s\mathcal S$, we have the function complex $\Map_{s\mathcal S}(X,Y)$, where each $n$-simplex is the set of simplicial maps $X\times\Delta^n\to Y$.

Let $F(k)$ be the simplicial space defined by $[n]\mapsto \Delta([n],[k])$. where the set of morphisms $\Delta([n],[k])$ in $\mathcal S$ is taken as a discrete simplicial set. $F(k)$ is a $k$-th space functor, in the sense that there is an isomorphism
$$\Map_{s\mathcal S}(F(k),X)\cong X_k$$
of simplicial sets which is natural with respect to $d^i:F(k)\to F(k+1)$ and $s^i:F(k)\to F(k-1)$.
Let $\partial F(k)$ be the largest subobject of $F(k)$ not containing the identity map $\iota:[k]\to[k]$. It can easily be seen that $\partial F(k)$ is generated by the faces $d_i\iota\in\Delta([k-1],[k])$. We denote by $\partial X_k$ the mapping space $\Map_{s\mathcal S}(\partial F(k),X)$. The inclusion $\partial F(k)\hookrightarrow F(k)$ induces a map $X_k\to\partial X_k$.

\begin{prop}
The category of simplicial spaces $s\cal S$ is catesian closed, that is, for $X,Y\in s\mathcal S$, there exists an internal hom-object $Y^X$ with a natural isomorphism
$$s\mathcal S(X\times Y,Z)\cong s\mathcal S(X,Z^Y).$$
\end{prop}
\begin{proof}
Let $(Y^X)_k=\Map_{s\mathcal S}(X\times F(k),Y)$.
\end{proof}

\subsection{Reedy model category structure on $s\mathcal S$}
\begin{thm}
There exists a model structure on $s\mathcal S$ where $f:X\to Y$ is a
\begin{enumerate}
\item weak equivalences if $f_k:X_k\to Y_k$ are degree-wise weak equivalences;
\item cofibrations if $f_k$ are degree-wise cofibrations;
\item fibrations if the induced map
\begin{equation} \label{Reedyfib}
X_k\to Y_k\times_{\partial Y_k}\partial X_k
\end{equation}
are fibrations.
\end{enumerate}
\end{thm}
\begin{proof}
See \cite[IV.3.2]{gj}.
\end{proof}

This is called the {\bf Reedy model structure} on the category of simplicial spaces. It is given by the Reedy model structure construction on the model category $\mathcal S$ and the Reedy category $\Delta$. Note that all simplicial spaces are cofibrant and a simplicial space $X$ is Reedy fibrant iff
$$X_k\to\partial X_k,\qquad k\ge 0.$$

As with the standard model structure on the category of simplicial sets, the Reedy model category structure is cofibrantly generated \cite{dhk}, that is, there exist sets of generating cofibrations and generating trivial cofibrations such that trivial fibrations (fibrations, respectively) are characterised by the right lifting property with respect to the set of generating cofibrations (generating trivial cofibrations). The generating cofibrations are
$$\partial F(k)\times \Delta^l\sqcup_{\partial F(k)\times\partial\Delta^l}F(k)\times\partial\Delta^l\to F(k)\times\Delta^l,\qquad k,l\ge0$$
and the generating trivial cofibrations are
$$\partial F(k)\times \Delta^l\sqcup_{\partial F(k)\times\Lambda^l_t}F(k)\times\Lambda^l_t\to F(k)\times\Delta^l,\qquad k\ge0,0\le t\le l.$$

\begin{defin}
A model category structure on $\cal C$ is compatible with cartesian closure if for any cofibrations $i:A\to B$, $j:C\to D$ and fibration $p:X\to Y$, either (and hence both) of the equivalent characterisations hold:
\begin{enumerate}
\item the induced map $A\times D\sqcup_{A\times C}B\times C\to B\times D$ is a cofibration, which is trivial if either $i$ or $j$ is; or
\item the induced map $X^B\to X^A\times_{Y^A}Y^B$ is a fibration, which is trivial if either $i$ or $p$ is.
\end{enumerate}
\end{defin}

\begin{prop}
$s\mathcal S$ with the Reedy model structure is compatible with cartesian closure.
\end{prop}
\begin{proof}
It suffices to check condition (i). This holds since cofibrations and weak equivalences are defined degree-wise and the standard model structure on $\mathcal S$ is compatible with cartesian closure (in $\mathcal S$, the induced map in (i) is an anodyne extension).
\end{proof}

In particular, this implies that given any cofibration (inclusion) $A\hookrightarrow B$ and $X$ fibrant, we have a fibration $\Map_{s\mathcal S}(B,X)\to\Map_{s\mathcal S}(A,X)$, which is trivial if $A\hookrightarrow B$ is.

Recall the definition of a proper model category.
\begin{defin}
A model category is proper if
\begin{enumerate}
\item the pushout of a weak equivalence along a cofibration is a weak equivalence; and
\item the pullback of a weak equivalence along a fibration is a weak equivalence.
\end{enumerate}
\end{defin}

\begin{prop}
$s\mathcal S$ is a proper model category.
\end{prop}
\begin{proof}
$\mathcal S$ is a proper model category. Since cofibrations and weak equivalences are degree-wise, condition (i) follows trivially.
Condition (ii) follows from the fact that under the Reedy model category structure, if all objects are cofibrant, then all fibrations are also degree-wise fibrations. Indeed, since $\partial F(k)$ is cofibrant and the Reedy model structure is compatible with cartesian closure, the cofibration $\emptyset\to\partial F(k)$ and the fibration $X\to Y$ induces a fibration of internal hom-objects
$$X^{\partial F(k)}\to Y^{\partial F(k)}\times_{Y^\emptyset}X^\emptyset\cong Y^{\partial F(k)}$$
This thus induces a fibration on the 0-spaces $\partial X_k\to\partial Y_k$. Pulling back along $Y_k\to\partial Y_k$ and composing with (\ref{Reedyfib}), we get a fibration
$$X_k\to\partial X_k\times_{\partial Y_k}Y_k\to Y_k.$$
\end{proof}

We state a property of proper model categories:

\begin{prop}\label{hompullback}
Let $\cal C$ be a proper model category. Then, the pushout along a cofibration is a homotopy pushout and the pullback along a fibration is a homotopy pullback.
\end{prop}
\begin{proof}
See \cite[Prop A.2.2.4]{htt}.
\end{proof}

\section{Segal spaces}
In this section, we will construct our first model of $(\infty,1)$-categories, the complete Segal spaces. Complete Segal spaces have an explicit homotopy structure, which Rezk described as the study of homotopy theory of homotopy theories. For most of this section, we will give explicit constructions following Rezk's paper \cite{rezk}.

\subsection{The Segal conditon}
The Segal condition is a modification of the $\Delta$-space defined by Graeme Segal, which is a simplicial space $X$ in which $X_n$ is naturally weakly equivalent to $(X_1)^n$. In the Segal condition, we allow the $0$-space to be more than a single point.

For $0\le i<k$, let $\alpha^i:[1]\to [k]$ be the map sending $[0,1]\mapsto [i,i+1]$. Let $G(k)\subset F(k)$ be the simplicial subspace generated by $\alpha^i\in F(k)_1$. Equivalently, $\alpha^i$ induces a map $F(1)\to F(k)$ and we define $G(k)$ to be
$$G(k)=\cup_{i=0}^{k-1}\alpha^iF(1)\subset F(k).$$
The inclusion $\phi^k:G(k)\hookrightarrow F(k)$ induces a map $\phi_k=\Map_{s\mathcal S}(\phi^k,X):\Map_{s\mathcal S}(F(k),X)\cong X_k\to\Map_{s\mathcal S}(G(k),X)$. We can check that
$$\Map_{s\mathcal S}(G(k),X)\cong X_1\times_{X_0}\cdots\times_{X_0}X_1=\lim(X_1\xrightarrow{d_0}X_0\xleftarrow{d_1}X_1\xrightarrow{d_0}\cdots\xrightarrow{d_0}X_0\xleftarrow{d_1}X_1)$$
with $k$ copies of $X_1$.

\begin{defin}
We say that a simplicial space $X$ satisfies the Segal condition if
\begin{equation} \label{segal}
\phi_k:X_k\to X_1\times_{X_0}\cdots\times_{X_0}X_1
\end{equation}
is a weak equivalence for each $k\ge 2$.
\end{defin}

Therefore, two simplicial spaces satisfying the Segal condition are defined up to weak equivalence by their 0-th and 1-st spaces.

We can define a simplicial enriched structure associated to a simplicial space satisfying the Segal condition:
\begin{defin}\label{simpcat}
Let $W$ be a simplicial space. Define $\Ob W=(W_0)_0$ to be the vertices of $W_0$ and for any $x,y\in\Ob W$, we define $\map_W(x,y)$ to be the fiber of the map $(d_1,d_0):W_1\to W_0\times W_0$ at the point $(x,y)\in W_0\times W_0$. The identity map is defined to be $\id_x=s_0x\in\Map_W(x,x)$.
\end{defin}
In general, for a simplicial space $W$ satisfying the Segal condition, $\map_W(x,y)$ is not fibrant, so we cannot define a homotopy equivalence on the simplicial set. However, requiring $\map_W(x,y)$ to be fibrant for all $(x,y)$ (eg. by requiring $(d_1,d_0):W_1\to W_0\times W_0$ to be a fibration) is still not sufficient. We also want the following condition: we want a homotopy relation that is well-defined up to changing of the end points by a path in $(W_0)_1$, that is, if $[x],[y]\in\pi_0(W_0)$ are the path-components of $x$ and $y$, then we want $\map_W([x],[y])$ to be defined in some way and a homotopy equivalence relation on it.

It turns out that an appropriate condition to impose is Reedy fibrancy.
\begin{defin}
A Segal space is a Reedy fibrant simplicial space satisfying the Segal condition.
\end{defin}
Note that since a Segal space $W$ is fibrant and $G(k)\subset F(k)$ is a cofibration, the map $\phi_k$ is a fibration for all $k$. Similarly, $d_0,d_1:W_1\to W_0$ are fibrations, hence $(d_1,d_0):W_1\to W_0\times W_0$ is a fibration and $\map_W(x,y)$ are fibrant. Furthermore, $W_1\times_{W_0}\cdots\times_{W_0}W_1$ is a homotopy fibre product.

In particular, $W_0$ is fibrant as $W$ is fibrant, and compositions of the fibrant maps in the previous paragraph gives us that $W_k$ is fibrant for all $k$. Hence, we can view a Segal space as a Reedy fibrant simplicial object in the category of topological spaces satisfying the Kan condition. This view is more convenient for geometrical examples. For a complete description, see \cite{lurietft}.

\begin{eg}
Every discrete simplicial space ($W_k$ is discrete for each $k$) is Reedy fibrant. Hence, it is a Segal space iff it satisfies the Segal condition. Indeed, if a discrete simplicial space satisfies the Segal condition, then $\phi_k$ will be isomorphisms.
\end{eg}

\subsection{Examples of Segal spaces}\label{egss}
\subsubsection{Classification diagram of categories}
We denote by $\mathrm{nerve}(C)$ the nerve of a category $C$, that is, the simplicial set with $n$-simplices given by a chain of composable morphisms
$$c_0\to\ldots\to c_n.$$
\begin{prop}\label{nerve}
$\mathrm{nerve}([n])=\Delta^n$. For any categories $C$ and $D$, there are natural isomorphism $\mathrm{nerve}(C\times D)\cong\mathrm{nerve}(C)\times\mathrm{nerve}(D)$ and $\mathrm{nerve}(D^C)\cong\mathrm{nerve}(D)^{\mathrm{nerve}(C)}$. The nerve functor gives a full embedding $\mathrm{nerve}:{\cal C}at\to\mathcal S$.
\end{prop}

Let $C$ be a category and $W\subset C$ a subcategory such that $\Ob W=\Ob C$. 

\begin{defin}
Let $(C,W)$ be a category with its subcategory of weak equivalences. We say that a morphism $f$ is a {\bf weak equivalence} if $f\in W$.

Let $D$ be any other category. For any two functors $f,g\in C^D$, we say that a natural transformation $f\xrightarrow{\alpha}g$, we say that $\alpha$ is a {\bf weak equivalence} if $\alpha d \colon f(d) \to g(d)$ is a weak equivalence for all $d\in\Ob D$. Let $\mathrm{we}(C^D)\subset C^D$ be the subcategory of all weak equivalences.

The {\bf classification diagram} of $(C,W)$ is defined to be the simplicial space $N(C,W)$ where
\[
N(C,W)_m=\mathrm{nerve}\,\mathrm{we}(C^{[m]}).
\]
\end{defin}
It is convenient to view an $n$-simplex of $N(C,W)_m$ as a diagram
\[
\begin{matrix} c_{00}&\to&\cdots&\to&c_{0m}\\\downarrow&&&&\downarrow\\\vdots&&&&\vdots\\\downarrow&&&&\downarrow\\c_{n0}&\to&\cdots&\to&c_{nm}
\end{matrix}
\]
where the vertical arrows are weak equivalences.

We consider some special cases:
\begin{eg}\label{nerveeg}
\begin{enumerate}
\item Let $C_0\subset C$ be the subcategory consisting of all objects and only the identity morphisms. Then, $\mathrm{discnerve}\,C=N(C,C_0)$ is known as the {\bf discrete nerve}. In particular $\mathrm{discnerve}([n])=F(n)$. However, note that equivalent categories may not give weakly-equivalent discrete nerves.

\item Let $\mathrm{iso}\,C\subset C$ be the subcategory consisting of all objects and all invertible morphisms in $C$ (i.e. the maximal subgroupoid of $C$). The {\bf classifying diagram} of $C$ is defined to be $NC=N(C,\mathrm{iso}\,C)$.
\end{enumerate}
\end{eg}

\begin{prop}
The classifying diagrams $\mathrm{discnerve}\,C$ and $NC$ of a category $C$ are Segal spaces.
\end{prop}
\begin{proof}
(i) Note that $(\mathrm{discnerve}\,C)_m\cong(\mathrm{nerve}\,C)_m$ are discrete simplicial sets, so it is clear that $\mathrm{discnerve}\,C$ satisfies the Segal condition. Furthermore, all discrete simplicial spaces are Reedy fibrant.

(ii) Let $I[n]$ denote the category with $n+1$ objects with a unique isomorphism between each pair of objects. An $n$-simplex in $NC_m$ can thus be seen as a functor $[m]\times I[n]\to C$. It is then easy to see that there is a natural isomorphism
$$NC_m\cong NC_1\times_{NC_0}\cdots\times_{NC_0}NC_1.$$
To show that $NC$ is Reedy fibrant, we need to show that the maps $l_m:NC_m\to\partial NC_m$ are fibrations for $m\ge0$. They are easy to check using the above representation of $n$-simplices in $NC_m$. Indeed, $l_m$ is an isomorphism for $m\ge 3$.
\end{proof}

We obtain a result similar to Prop.~\ref{nerve}:
\begin{prop}\label{nerveprop}
Let $C$ and $D$ be categories. There are natural isomorphisms $N(C\times D)\cong NC\times ND$ and $N(D^C)\cong ND^{NC}$. More generally, given $\mathrm{iso}\,D\subset W\subset D$ a subcategory,
\begin{equation} \label{nfuncat}
N(D^C,\mathrm{we}(D^C))\cong N(D,W)^{NC}\cong N(D,W)^{\mathrm{discnerve}\,C}.
\end{equation}
The functor $N:{\cal C}at\to s\mathcal S$ is a full embedding of categories. $F:C\to D$ is an equivalence of categories iff $NF$ is a weak equivalence of simplicial spaces.
\end{prop}
\begin{proof}
See \cite[Thm 3.7, Prop 3.11]{rezk}.
\end{proof}

\subsubsection{Classification space of a closed model category}
Let $C=M$ be a closed model category and $W\subset M$ be its subcategory of weak equivalences. As noted above, we have a simplicial space $N(C,W)$. $N(C,W)$ is in general not Reedy fibrant. However, we can take a functorial Reedy fibrant replacement of it, for example, by the small object argument. Note that taking a Reedy fibrant replacement does not change the homotopy type of the spaces in each degree.

\begin{defin}
The classification space of a simplicial close model category $(M,W)$ is a functorial Reedy fibrant replacement $N^f(M)$ of $N(M,W)$.
\end{defin}

As most of the model categories we are interested in are not small, we need to be take care of some set theoretical considerations. However, we can always work in a larger universe, so $N^f(M)$ may not be in the same universe as $M$.

\begin{thm}
$N^f(M)$ is a Segal space.
\end{thm}
\begin{proof}
See \cite[Thm 8.3]{rezk}.
\end{proof}

Note that any category $C$ with finite limits and colimits can be given a model structure in which the weak equivalences are exactly the isomorphisms. In this case, $N(C,W)=N(C,\mathrm{iso}\,C)$ is already a Segal space, so $N^f(C)=NC$. However, in general, it is difficult to compute the Reedy fibrant replacement of a simplicial space, and thus the classifying space. Later, we will show another way to obtain the classification space through a localisation functor.

Given a small indexing category $I$, we can define a subcategory of weak equivalence $\mathrm{we}(M^I)\subset M^I$. (\ref{nfuncat}) and the Reedy fibrant replacement functor induces a natural map
$$f:N(M^I,\mathrm{we}(M^I))\cong N(M,W)^{\mathrm{discnerve}\,I}\to N^f(M)^{\mathrm{discnerve}\,I}.$$
In general, the map $f$ is not a weak equivalence, but a result by Dwyer and Kan says that, in some special cases, for example if $M$ is cofibrantly generated, $f$ is a weak equivalence. In that case, we see that the homotopy type of the classification diagram of the functor category $M^I$ is entirely determined by the homotopy type of $M$. More precisely, we state the following theorem by Dwyer and Kan.
\begin{thm}
Suppose $J$ is a small indexing category and $M=\mathcal S^J$, then the map $f:N(M^I,\mathrm{we}(M^I))\to N^f(M)^{\mathrm{discnerve}\,I}\cong N^f(M)^{NI}$ is a weak equivalence. In particlar, it induces a weak equivalence of Segal spaces $N^f(M^I)\to N^f(M)^{NI}$.
\end{thm}
\begin{proof}
See \cite[Thm 8.11]{rezk}.
\end{proof}


\subsection{Homotopy theory in Segal spaces and categories}
Recall the construction of a simplicial enriched structure associated to a Segal space, given in Def. \ref{simpcat} (It may not a simplicial enriched category since associativity may not be strict). That construction allows us to construct the homotopy category associated to the Segal space.

We begin by defining homotopic morphisms.
\begin{defin}
Let $W$ be a Segal space and $x,y\in\Ob W$. Two maps $f,g\in\map(x,y)$ are homotopic if they lie in the same path component of $\map(x,y)$, i.e., $[f]=[g]\in\pi_0\map(x,y)$.
\end{defin}

In general, given $k+1$ objects $x_0,\to,x_k\in\Ob W$, we can define $\map_W(x_0,\to,x_k)$ to be the fibre of the fibration $(\alpha_0,\ldots,\alpha_n):W_k\to W_0^k$ at the point $(x_0,\ldots,x_k)$, where $\alpha_i$ is the map induced by $\alpha^i:[0]\to[k]$ taking 0 to $i$. The fibre of the commutative diagram

\centerline{\xymatrix{W_k\ar[rr]^{\sim}_{\phi_k}\ar[dr]_{(\alpha_0,\ldots,\alpha_n)}&&W_1\times_{W_0}\cdots\times_{W_0}W_1\ar[dl]\\
&W_0^{k+1}}}

induces a trivial fibration
$$\phi_k:\map(x_0,\ldots,x_k)\to\map(x_0,x_1)\times\cdots\times\map(x_{n-1},x_n).$$

We can thus define composition of maps up to homotopy.
\begin{defin}
If $f\in\map(x,y)$ and $g\in\map(y,z)$, we define $g\circ f=d_1h$ for some $h\in\map(x,y,z)$ such that $\phi_2(h)\sim(f,g)$ are homotopic.
\end{defin}

\begin{prop}
For each Segal space $W$, we have an associated homotopy category $\Ho W$ where $\Ob\Ho W=\Ob W$ and $\map_{\Ho W}(x,y)=\pi_0\map_W(x,y)$.
\end{prop}
\begin{proof}
It suffices to show that $f\circ(g\circ h)\sim(f\circ g)\circ h$ and $f\circ\id\sim f\sim\id\circ f$. See \cite[Prop 5.4]{rezk} for details.
\end{proof}

\begin{eg}
Let $C$ be a category. Then, $\Ob NC=\Ob C$ and $\map_{NC}(x,y)\cong\hom_C(x,y)$ is the discrete simplicial set generated by elements of $\hom_C(x,y)$. Thus, $\Ho NC\cong C$.
\end{eg}

Now, we define the notion of homotopy equivalence. Let $Z(3)=\mathrm{discnerve}(0\to2\leftarrow1\to3)\subset F(3)$. It induces a fibration $W_3\to\Map_{s\mathcal S}(Z(3),W)\cong W_1\times_{W_0}W_1\times_{W_0}W_1$.

\begin{prop}
Let $g\in\map_W(x,y)$. The following statements are equivalent:
\begin{enumerate}
\item There exist $f,h\in\map_W(y,x)$ such that $g\circ f\sim\id_y$ and $h\circ g\sim\id_y$.
\item $(\id_x,g,\id_y)\in\Map_{s\mathcal S}(Z(3),W)$ admits a lift to some $H\in W_3$.
\end{enumerate}
\end{prop}
\begin{proof}
Easy check.
\end{proof}

If $g$ satisfies either of the above equivalent statements, we call $g$ a {\bf homotopy equivalence}.

\begin{lemma}
If $g\in (W_1)_0$ is a vertex connected by a path in $(W_1)_1$ to a vertex $g'$, then $g$ is a homotopy equivalence iff $g'$ is one.
\end{lemma}
\begin{proof}
See \cite[Lemma 5.8]{rezk}.
\end{proof}

\begin{defin}
We can thus define the space of homotopy equivalence to be the components $W_{\mathrm{hoequiv}}\subset W_1$ of homotopy equivalences.
\end{defin}
Note that $s_0:W_0\to W_1$ factors through $W_{\mathrm{hoequiv}}$ since $s_0x=\id_x\in W_{\mathrm{hoequiv}}$ for all $x\in W_0$.

\subsection{Complete Segal spaces}
We have now defined Segal spaces and their homotopy categories. However, as we will see in the next section, there are too many Segal spaces with respect to other models of $(\infty,1)$-categories. See Example \ref{nervecomplete} below.
We define the following:
\begin{defin}
A complete Segal space is a Segal space such that the map $s_0:W_0\to W_{\mathrm{hoequiv}}$ is a weak equivalence.
\end{defin}

We want to find an object in $s\mathcal S$ that represents the functor $W\mapsto W_{\mathrm{hoequiv}}$, at least up to weak equivalence. Let $E=\mathrm{discnerve}(I[1])$. We have the following theorem:
\begin{thm}\label{rephoequiv}
The map $\Map_{s\mathcal S}(E,W)\to W_1$ induced by the inclusion $F(1)\hookrightarrow E$ factors through $W_{\mathrm{hoequiv}}\subset W_1$, and induces a weak equivalence $\Map_{s\mathcal S}(E,W)\to W_{\mathrm{hoequiv}}$.
\end{thm}
\begin{proof}
The proof is technical. See \cite[Thm 6.2]{rezk}.
\end{proof}

Some corollaries of the theorem include
\begin{cor}\label{rephoequivcor}
Let $W$ be a Segal space. The following are equivalent:
\begin{enumerate}
\item $W$ is a complete Segal space.
\item The map $W_0\to\Map_{s\mathcal S}(E,W)$ induced by $E\to F(0)$ is a weak equivalence.
\item For each pair $x,y\in\Ob W$, the fibre $\mathrm{hoequiv}(x,y)$ of the fibration $W_{\mathrm{hoequiv}}\xrightarrow{(d_1,d_0)} W_0\times W_0$ is naturally weak equivalent to the space of paths in $W_0$ from $x$ to $y$.
\end{enumerate}
\end{cor}
\begin{proof}
$(i)\Leftrightarrow(ii)$ is clear from Thm.~\ref{rephoequiv}.

$(ii)\Leftrightarrow(iii)$: consider the composition $\Delta:W_0\xrightarrow{s_0}W_{\mathrm{hoequiv}}\xrightarrow{(d_1,d_0)} W_0\times W_0$. Since $(d_1,d_0)$ is a fibration, $\mathrm{hoequiv}(x,y)$ is actually a homotopy fibre (Prop.~\ref{hompullback}). The spcae of paths in $W_0$ from $x$ to $y$ is the homotopy fibre of $\Delta$.
\end{proof}

\begin{cor}
Let $\Ob W/\sim$ denote the set of homotopy equivalence classes of objects in $\Ho W$. If $W$ is a complete Segal space, then $\pi_0W_0\cong\Ob W/\sim$.
\end{cor}
\begin{proof}
Follows immediately from (iii) of the previous corollary.
\end{proof}

\begin{eg}
The classifying diagram $N(C)$ for a category $C$ and the classifying space $N^f(M)$ for a simplicial closed model category $M$ are complete Segal spaces. See \cite{rezk} for details of the proofs. $\mathrm{discnerve}\,(C)$ is not complete.
\end{eg}

\begin{defin}
Let $W$ be a Segal space, we define the completion of $W$ to be a complete Segal space $\hat W$ with a map $i_W:W\to\hat W$ which is universal among all maps from $W$ to a complete Segal space.
\end{defin}

\begin{prop}
There exists a completion functor given by $i_W:W\to\hat W$ on the category of Segal spaces.
\end{prop}
\begin{proof}
Let $E(m)=\mathrm{discnerve}(I[m])$. For each $n \ge 0$, we can define a simplicial set $\tilde W_n=\diag([m]\mapsto (W^{E(m)})_n \cong \Map_{s\mathcal S}(E(m)\times F(n),W))$ where the diagonal map $\diag \colon s\mathcal S\cong\mathcal S^{(2)}\to\mathcal S$ is that induced by $[n]\mapsto[n]\times[n]$. The face and degeneracy maps induced from $d^i \colon F(n)\to F(n+1)$ and $s^i \colon F(n)\to F(n-1)$ gives us a simplicial space $\tilde W$ with a natural map $W\to \tilde W$. $\hat W$ is defined to be a functorial Reedy fibrant replacement of $\tilde W$, thus inducing a map $i_W:W\to\tilde W\to\hat W$.

For the details of the proof, see \cite{rezk}.
\end{proof}

In general, the above construction of the completion of a Segal space is not easy to understand.

\begin{eg}\label{nervecomplete}
Given a category $C$, $\mathrm{discnerve}\,C$ is a Segal space, but not complete. Its completion $\widehat{\mathrm{discnerve}\,C}=NC$. As mentioned in Example \ref{nerveeg}, equivalent categories may give rise to non-weakly equivalent discrete nerves, but we see that the completions are weakly equivalent iff the categories are equivalent (Prop.~\ref{nerveprop}).
\end{eg}

\subsection{Closed model category structures related to Segal spaces}
In this section, we will introduce two other model category structures on the category of simplicial spaces $s\mathcal S$. The fibrant-cofibrant objects will be the Segal spaces and the complete Segal spaces in the two structures respectively. We will also show the relationship between weak equivalences in these two structures and with Reedy weak equivalences.

\begin{thm}
There exists a closed model category structure on $s\mathcal S$ with the following properties:
\begin{enumerate}
\item The cofibrations are the monomorphisms.
\item The weak equivalences are maps $f$ such that $\Map_{s\mathcal S}(f,W)$ is a weak equivalence for all Segal spaces $W$.
\item The fibrations are the maps that satisfy the right lifting property with respect to all trivial cofibrations.
\end{enumerate}
This is called the {\bf Segal space model category structure} on $s\mathcal S$, an is denoted as ${\cal SS}$. This model structure is compatible with the cartesian closed standard model structure on $\mathcal S$. All objects are cofibrant and the fibrant objects are precisely the Segal spaces. A Reedy weak equivalence between two objects $X,Y$ is a weak equivalence in ${\cal SS}$ and the converse is true if $X,Y$ are Segal spaces.
\end{thm}
\begin{proof}
${\cal SS}$ is the left Bousfield localisation of the Reedy model category structure with respect to the set of maps $S=\{G(k)\hookrightarrow F(k)\}$. See \cite[Thm 7.1]{rezk} for the proof of the compatibility with cartesian closure.
\end{proof}

\begin{thm}
There exists a closed model category structure on $s\mathcal S$ with the following properties:
\begin{enumerate}
\item The cofibrations are the monomorphisms.
\item The weak equivalences are maps $f$ such that $\Map_{s\mathcal S}(f,W)$ is a weak equivalence for all complete Segal spaces $W$.
\item The fibrations are the maps that satisfy the right lifting property with respect to all trivial cofibrations.
\end{enumerate}
This is called the {\bf complete Segal space model category structure} on $s\mathcal S$, an is denoted as ${\cal CSS}$. This model structure is compatible with the cartesian closed standard model structure on $\mathcal S$. All objects are cofibrant and the fibrant objects are precisely the complete Segal spaces. A Reedy weak equivalence between two objects $X,Y$ is a weak equivalence in ${\cal CSS}$ and the converse is true if $X,Y$ are complete Segal spaces.
\end{thm}
\begin{proof}
By Cor.~\ref{rephoequivcor}, we see that ${\cal CSS}$ is the left Bousfield localisation of ${\cal SS}$ with respect to the map $E\to F(0)$. See \cite[Thm 7.2]{rezk}for the proof of the compatibility with cartesian closure.
\end{proof}

Let $\Ho{\cal SS}$ and $\Ho{\cal CSS}$ be the homotopy categories associated to the two model structures respectively, and $\Ho{\cal SS}_{cf}$ and $\Ho{\cal CSS}_{cf}$ denote the respective full subcategories of fibrant-cofibrant objects, namely the Segal spaces and the complete Segal spaces respectively.

Note that for any model category $M$, the inclusion $M_{cf}\subset M$ induces an equivalence of homotopy categories $\Ho M_{cf}\cong\Ho M$. This, in particular, implies that small homotopy limits and colimits exist in the subcategory $M_{cf}$.

An immediate consequence of the compatibility of these model structures with cartesian closure is that if $W$ is a (complete) Segal space and $X$ is a simplicial space, then $W^X$ is a (complete) Segal space.

To understand the relationship between the two model category structures, we introduce the notion of Dwyer-Kan equivalence.
\begin{defin}
A map $f:U\to V$ between two Segal spaces is a Dwyer-Kan equivalence if
\begin{enumerate}
\item the induced map $\Ho f:\Ho U\to\Ho V$ is an equivalence of categories; and
\item for each pair of objects $x,x'\in U$, the induced function $\map_U(x,x')\to\map_V(fx,fx')$ is a weak equivalence.
\end{enumerate}
An equivalent formulation of condition (i) is
\begin{enumerate}
\item the induced map $\Ob U/\sim\to\Ob V/\sim$ is a bijection on the equivalence classes of objects.
\end{enumerate}
\end{defin}
We have the following theorem
\begin{thm}
Let $f:U\to V$ be a map of Segal spaces. Then $f$ is a Dwyer-Kan equivalence iff $f$ is a weak equivalence in ${\cal CSS}$. If, in addition, $U$ and $V$ are complete Segal spaces, then $f$ is a Dwyer-Kan equivalence iff $f$ is a Reedy weak equivalence.
\end{thm}
\begin{proof}
See \cite[Thm 7.7]{rezk}.
\end{proof}

The proof of the above theorem relies on the following proposition, which we will state as it is also of interest on its own.
\begin{prop}\label{iw}
The completion functor $i_W:W\to\hat W$ is a Dwyer-Kan equivalence and a weak-equivalence in ${\cal CSS}$.
\end{prop}

\begin{proof}
See \cite[Sec.~14]{rezk}.
\end{proof}

\begin{cor}
If $f:U\to V$ map of Segal spaces is a weak equivalence in ${\cal SS}$, then $\hat f:\hat U\to\hat V$ is a weak equivalence in ${\cal CSS}$. Thus, the completion functor induces a well-defined functor between the homotopy categories $i:\Ho{\cal SS}_{cf}\to\Ho{\cal CSS}_{cf}$.
\end{cor}
\begin{proof}
This follows immediately from Prop.~\ref{iw} and the fact that a weak equivalence in ${\cal SS}$ is a weak equivalence in ${\cal CSS}$.
\end{proof}

\subsection{Classifying diagrams: another perspective}\label{sslocal}
To end of this section, we return to the example of the classifying diagram of a category $C$ with weak equivalences $W$. The ideas for this section are derived from the notes of a talk by To\"en \cite{toentalksegal}. The following diagram of categories
\begin{equation} \label{seglocalpushout}
\xymatrix{\coprod_{f\in W}[1]\ar[r]^{\sqcup i}\ar[d]&\coprod_{f\in W}I[1]\\C}
\end{equation}
induces a pushout square in $s\mathcal S$ of the classifying diagrams
\[
\xymatrix{
\coprod_{f \in W} F(1)\ar[r]^{\sqcup N(i)}\ar[d] & \coprod_{f\in W} N(I[1]) \ar[d] \\ NC\ar[r]&\tilde N(C,W)
}.
\]
Since $\sqcup i$ is a cofibration and all objects are cofibrant in ${\cal SCC}$, the square is also a homotopy pushout (see \cite[Prop A.2.2.4]{htt}). Since ${\cal SCC}_{cf}$ is closed under homotopy pushouts, there exists $\hat N(C,W)\in{\cal SCC}_{cf}$ such that $\hat N(C,W)$ is weakly equivalent to $\tilde N(C,W)$ and
$$\xymatrix{\coprod_{f\in W}F(1)\ar[r]^{\sqcup N(i)}\ar[d]&\coprod_{f\in W}N(I[1])\ar[d]\\NC\ar[r]&\hat N(C,W)}$$
is a homotopy pushout square in ${\cal SCC}_{cf}$.

If $W\subset\mathrm{iso}\,C$, then there exists a lift
$$\xymatrix{\coprod_{f\in W}F(1)\ar[r]^{\sqcup N(i)}\ar[d]&\coprod_{f\in W}N(I[1])\ar[d]\ar@{-->}[dl]_{\exists}\\NC\ar[r]&\hat N(C,W)},$$
so $\hat N(C,W)=NC$.

If $C=M$ is a closed model category with weak equivalences $W$, we want to show that $\hat N(C,W)\cong N^f(M)$, the classifying space we previously constructed.


\section{Segal categories}
We now introduce another model of $(\infty,1)$-categories. Segal categories were formally defined by Dwyer, Kan and Smith in \cite{dks}. They were used extensively and generalised to $n$-Segal categories by Hirschowitz and Simpson in their studies of $n$-stacks \cite{hs}. In this section, we will follow the ideas in \cite{hs} but refer to the work of Bergner \cite{bergner2} for more explicit constructions. 

\subsection{Segal precategories, categories and closed model structure}
\begin{defin}
A simplicial space $X$ is a Segal precategory if $X_0$ is discrete. Let ${\cal PC}at$ be the category of Segal pre-categories. A Segal precategory $X$ that satisfies the Segal condition (\ref{segal}) is called a Segal category.
\end{defin}

\begin{eg}
Let $C$ be any category, then $\mathrm{discnerve}\,C$ is a Segal category. Note that under the model category structure that we are going to impose on ${\cal PC}at$, $f:C\to D$ is an equivalence of categories iff $\mathrm{discnerve}\,f$ is a weak equivalence (unlike in ${\cal CSS}$).
\end{eg}

Every simplicial enriched category $C$ can be seen as a Segal precategory, which we will also denote by $C$, by setting $C_0=\Ob C$ and $C_n=\sqcup_{x,y\in\Ob C}(\map_C(x,y))_{n-1}$ with appropriate face and degeneracy maps. Note that, however, a Segal category may not be a simplicial category since associativity is not strict. To obtain a simplicial category, we will need to consider the category generated by the Segal category, which we will not define here.

Recall that given a simplicial space $X$, we can define $\Ob X$ and mapping spaces $\map_X(x,y)$ (Def.~\ref{simpcat}). We denote $x\sim y$ if $\map_X(x,y)\ne\emptyset$. However, $\sim$ may not be an equivalence relation. We denote by the same symbol the equivalence relation generated by $\sim$.

We have a well-defined notion of equivalence of Segal categories.
\begin{defin}
Let $f:A\to B$ be a morphism of Segal categories. We say that $f$ is a Dwyer-Kan equivalence if
\begin{enumerate}
\item the induced map $\Ob A/\sim\to\Ob B/\sim$ is surjective (we say that $f$ is essentially surjective); and
\item for any $x,y\in A$, the induced map $\map_A(x,y)\to\map_B(fx,fy)$ is a weak equivalence (we say that $f$ is fully faithful).
\end{enumerate}
\end{defin}
Note that injectivity in condition (i) follows from condition (ii).

To define the closed model structure on Segal precategories, we first need to construct a functor $L_C:{\cal PC}at\to{\cal PC}at$ that sends a precategory $X$ into a Segal category $L_CX$. Hirschowitz and Simpsons proved the existence of such a functor (indeed for Segal $n$-categories) in \cite{hs} but we shall give the explicit construction given by Bergner \cite{bergner2}.

We want to construct $L_CX$ as a functorial fibrant replacement of $X$ in the Segal space model category structure ${\cal SS}$, in such a way that $L_CX$ is still a precategory. We proceed by the small object argument. We have a set of generating trivial cofibrations in ${\cal SS}$ (obtained from the generating trivial cofibrations of the Reedy model structure by localisation):
$$F(k)\times\Lambda^l_t\sqcup_{G(k)\times\Lambda^l_t}G(k)\times\Delta^l\to F(k)\times\Delta^l,\qquad k\ge0,l\ge1,0\le t\le l$$
where $G(0)=\emptyset$. The fibrant replacement can be constructed as a colimit of the iterated pushout
$$\xymatrix{\coprod(F(k)\times\Lambda^l_t\sqcup_{G(k)\times\Lambda^l_t}G(k)\times\Delta^l)\ar[r]\ar[d]&X_i\ar[d]\\
\coprod(F(k)\times\Delta^l)\ar[r]&X_{i+1}}$$
where $X=X_0$ and the coproduct is taken over all maps $F(k)\times\Lambda^l_t\sqcup_{G(k)\times\Lambda^l_t}G(k)\times\Delta^l\to X_i$ with $k\ge 0$, $l\ge 1$ and $0\le t\le l$.

Note that the map on the zero space induced by a generating trivial cofibration is given by $[k]\times\Lambda^l_t\sqcup[k]\times\Delta^l\cong [k]\times\Delta^l\xrightarrow{\id}[k]\times\Delta^l$ for $k> 0$ and $\Lambda^l_t\to\Delta^l$ for a $k=0$.

Thus, the Segal space thus obtained may not have a discrete 0-space, and is thus not a Segal category. To fix this, we exclude the maps with $k=0$ and consider the iterated pushout with respect to the coproduct over the subset of maps with $k>0$. Let $L_CX$ be the colimit of this sequence of pushouts.

\begin{prop}
$L_CX$ as defined above is a functorial fibrant replacement of $X$, so $L_C:{\cal PC}at\to{\cal PC}at$ is a well-defined functor taking a precategory to a Segal category that is also a Segal space.
\end{prop}

By the small object argument, to check that this is indeed a fibrant replacement of $X$ in ${\cal SS}$, it suffices to check that $L_CX$ satisfies the RLP with respect to all generating trivial cofibrations with $k=0$. This is equivalent to checking that the lift exists in the following diagram
$$\xymatrix{\Lambda^l_t\ar[r]\ar[d]&\Map_{s\mathcal S}(F(0),L_CX)\cong (L_CX)_0\\\Delta^l\ar@{-->}[ur]}.$$
This is true since $L_CX$ is a discrete simplicial set and hence a Kan complex.\qed

We remark that in the case where $f:X\to Y$ is a map between two Segal categories which are also Segal spaces, then the two definitions of Dwyer-Kan equivalence are equivalent.

We are now ready to define a closed model category structure on precategories.

\begin{thm}
There exists a closed model category structure on ${\cal PC}at$ in which
\begin{enumerate}
\item the cofibrations are precisely the monomorphisms;
\item the weak equivalences are precisely the maps $f:X\to Y$ such that $L_Cf:L_CX\to L_CY$ is a Dwyer-Kan equivalence; and
\item the fibrations are maps that satisfy the right lifting property with respect to all trivial cofibrations.
\end{enumerate}
We denote ${\cal PC}at$ equipped with this model category structure ${\cal SC}$. The fibrant-cofibrant objects of this model structure are precisely the Reedy-fibrant Segal categories.
\end{thm}
\begin{proof}
See \cite[Thm 2.3]{hs} and \cite[Thm 5.1]{bergner2} for two different proofs of the existence of this model structure. See \cite[Cor 5.13]{bergner2} and \cite[Thm 3.2]{bergner3} for the proof of the last statement.
\end{proof}

Let $\mathrm{SeCat}\subset{\cal SC}$ and $\Ho\mathrm{SeCat}\subset\Ho\cal{SC}$ be the full subcategories of Segal categories. Since $\mathrm{SeCat}$ contains all fibrant-cofibrant objects, $\Ho\mathrm{SeCat}\cong\Ho\cal{SC}$ and in particular contains all small homotopy limits and colimits.

\subsection{Segal localisation}
We want to construct a localisation similar to that in Section \ref{sslocal}. Let $C$ be a category and $W\subset C$ be a subcategory of weak equivalences. Taking discrete nerves on the diagram of categories (\ref{seglocalpushout}), we obtain a pushout diagram in ${\cal SC}$:
$$\xymatrix{\coprod_{f\in W}F(1)\ar[r]\ar[d]&\coprod_{f\in W}E\ar[d]\\\mathrm{discnerve}\,C\ar[r]&\tilde C}.$$
Since the top arrow is a cofibration and all objects are cofibrant, it is a homotopy pushout square as well. Hence, there exists $L(C,W)\in\mathrm{SeCat}$ (indeed, we can even choose $L(C,W)\in{\cal SC}_{cf}$ the subcategory of Reedy-fibrant Segal categories) such that $L(C,W)$ is weakly equivalent to $\tilde C$ and the following diagram is a homotopy pushout square in $\mathrm{SeCat}$:
$$\xymatrix{\coprod_{f\in W}F(1)\ar[r]\ar[d]&\coprod_{f\in W}E\ar[d]\\\mathrm{discnerve}\,C\ar[r]&L(C,W)}.$$

Note that since the completion functor commutes with colimits and $\widehat{\mathrm{discnerve}\,C}\cong NC$, we have that $\widehat{L(C,W)}\cong\hat N(C,W)$.

If $W=\mathrm{iso}\,C$, we get $LC=\mathrm{discnerve}\,C$ as seen in Section \ref{sslocal}.

A result of Dwyer and Kan gives us a more explicit construction of $L(C,W)$.

\begin{thm}
Let $(C,W)$ be a category with a subcategory of weak equivalences. Then $L(C,W)$ is given by the hammock localisation $L^H(C,W)$.
\end{thm}
\begin{proof}
See \cite{dkcomputing}.
\end{proof}

Now, let $M$ be a simplicial closed model category, with weak equivalences $W$. Let $M_{cf}\subset M$ be the simplicial subcategory. Dwyer and Kan proved the following result:

\begin{thm}
Let $M$ be a closed simplicial model category. Then, $LM=L(M,W)\cong L^HM\cong M_{cf}$ is a weak equivalence.
\end{thm}
\begin{proof}
See \cite{dkfunction}.
\end{proof}

This theorem allows us to compute the localisation of simplicial model categories easily.
\begin{eg}
\begin{enumerate}
\item Let $M=\mathcal S$ be the category of simplicial sets with the usual model structure, then $M_{cf}$ is the subcategory of Kan complexes. $M_{cf}$ satisfies the Segal condition since homotopy is an equivalence relation on Kan complexes. Hence, $L\mathcal S$ gives us the theory of Kan complexes (or equivalently CW-complexes) as an $(\infty,1)$-category.
\item Let $M=\mathrm{Top}$ be the category of topological spaces, then $L\mathrm{Top}\cong\mathrm{Top}_{cf}$ is the subcategory of CW-complexes. Hence, we see that the localisation of $\mathcal S$ and $\mathrm{Top}$ give the same Segal category.
\item Let $M=s\mathcal S$ with the Reedy model structure. We thus get $Ls\mathcal S$ as the Segal category of all Reedy-fibrant simplicial spaces. If we equip the simplicial spaces with the (complete) Segal space model structure, we get $L{\cal SS}$ ($L{\cal CSS}$) the Segal category of all (complete) Segal spaces.
\item Let $M={\cal SC}$, then $L{\cal SC}$ is the Segal category of all Segal categories (up to some change in universe).
\end{enumerate}
\end{eg}
We see that Segal localisation gives us a $(\infty,1)$-category of $(\infty,1)$-categories. As in the case of complete Segal spaces, we have a strictification theorem.

\begin{thm}
Let $M$ be a cofibrantly generated simplicial model category and $I$ a small category, then we have a weak equivalence of Segal categories (Dwyer-Kan equivalence)
$$L(M^I)\cong L(M)^{\mathrm{discnerve}\,I}.$$
\end{thm}
\begin{proof}
See \cite{hs} and \cite{tv}.
\end{proof}

In particular, this implies that to compute small limits and colimits of (complete) Segal spaces or Segal categories, it suffices to compute in the larger categories of simplicial spaces or precategories. See To\"en and Vezzosi's paper \cite{tv} for more details.


\section{Comparison theorems}
In this section, we will state a number of comparison theorems between complete Segal spaces, Segal categories and other models of $(\infty,1)$-categories.

The main mechanism in proving the equivalence of two model categories is Quillen equivalence. Recall the following definitions:
\begin{defin}
A pair of adjoint functors $F:C\rightleftarrows D:G$ (with $\phi:\Hom_D(FX,Y)\xrightarrow{\sim}\Hom_C(X,GY)$) is a {\bf Quillen pair} if it satisfies one of the following equivalent conditions:
\begin{enumerate}
\item $F$ preserves cofibrations and $G$ preserves fibrations;
\item $F$ preserves cofibrations and trivial cofibrations; or
\item $G$ preserves fibrations and trivial fibrations.
\end{enumerate}
A Quillen pair $(F,G)$ is a {\bf Quillen equivalence} if, in addition, $f:FX\to Y$ is a weak equivalence in $D$ iff $\phi f:X\to GY$ is a weak equivalence in $C$.
\end{defin}
We also recall the theorem:
\begin{thm}
A Quillen pair $F:C\rightleftarrows D:G$ induces an adjoint pair of left and right total derived functors
$$LF:\Ho C\rightleftarrows\Ho D:RG$$
which is an equivalence of categories if $(F,G)$ is a Quillen equivalence. In this case, we also have an equivalence of full subcategories
$$LF:\Ho C_{cf}\rightleftarrows\Ho D_{cf}:RG.$$
\end{thm}
This implies that if we have a Quillen equivalence between two models of infinity categories, they have the same objects up to weak equivalences. Furthermore, weak equivalences in the full subcategory $M_{cf}\subset M$ of a model category is the same as homotopy equivalence, so they are defined up to homotopy equivalence.

We have the following diagram of Quillen equivalences.
$$\xymatrix{{\cal CSS}\ar@/^/[dr]&{\cal SC}\ar[l]\ar@/^/[d]&{\cal SC}'\ar[l]\ar[r]&s{\cal C}\\
&{\cal QC}\ar@/^/[ul]\ar@/^/[u]\ar[urr]}$$
An arrow in the diagram refers to the direction of the left adjoint. ${\cal QC}$ is the category of simplicial sets with the Joyal model category structure, $s{\cal C}$ is the category of simplicial enriched categories and ${\cal SC}'$ is another model structure on precategories.

We shall only present the Quillen equivalences among ${\cal CSS}$, ${\cal SC}$ and ${\cal QC}$. We refer to Bergner's paper \cite{bergner3} for the construction of the fibrant model category structure ${\cal SC}'$ on precategories (as opposed to the cofibrant structure ${\cal SC}$). It was constructed to proved the equivalence with $s{\cal C}$. The equivalence between ${\cal SC}$ and ${\cal SC}'$ is given by the identity functor. For the equivalence between simplicial categories and quasi-categories, we refer to \cite{joyal3}.

In this section, we will not present the proofs, but instead just sketch the construction of the functors or model structures.

\subsection{Some model category structures}
First we construct the three model category structures that we have yet to construct.

\begin{defin}
Let $X$ be a simplicial set, for any $x,y\in X_0$, we write $x\sim y$ if there exists $w\in X_1$ such that $d_1w=x$ and $d_0w=y$. we define $\tau_0X$ be the set of equivalence class in $X_0$ under the equivalence relation generated by $\sim$.

Let $X(x,y)$ be the fibre of the projection $X^I\xrightarrow{(d_1,d_0)} X\times X$.

Let $f:X\to Y$ be a map of simplicial sets, we say that $f$ is essentially surjective if the induced map $\tau_0X\to\tau_0Y$ is surjective. We say that $f$ is fully faithful if for every pair $x,y\in X_0$, the induced map $X(x,y)\to Y(fx,fy)$ is a weak equivalence (under the standard model category structure). We say that $f$ is a weak categorical equivalence if $f$ is both essentially surjective and fully faithful.
\end{defin}
\begin{thm}
There is a closed model category structure on $\mathcal S$, which we will denote as ${\cal QC}$, the quasi-category model structure, in which
\begin{enumerate}
\item the cofibrations are precisely the monomorphisms;
\item the weak equivalences are precisely the weak categorical equivalences; and
\item the fibrations are maps satisfying the right lifting property with respect to trivial cofibrations.
\end{enumerate}
The fibrant-cofibrant objects in this model structure are precisely the quasi-categories.
\end{thm}

\begin{proof}
See \cite{joyal2}.
\end{proof}

Next we define a closed model category structure on simplical categories.
\begin{thm}
There is a closed model category structure on $s{\cal C}$, in which
\begin{enumerate}
\item the weak equivalences are the Dwyer-Kan equivalences;
\item the fibrations are maps $F:C\to D$ satisfying:
\begin{itemize}
\item for any $x,y\in C$, the induced map $\map_C(x,y)\to\map_D(Fx,Fy)$ is a fibration of simplicial sets;
\item for any $x_1\in C$, $y\in D$ and homotopy equivalence $e:Fx_1\to y$ in $D$, there exists $x_2\in C$ and homotopy equivalence $d:x_1\to x_2$ such that $Fd=e$.
\end{itemize}
\item the cofibrations are maps satisfying the left lifting property with respect to all trivial fibrations.
\end{enumerate}
\end{thm}

\begin{proof}
See \cite{bergner4}
\end{proof}

\subsection{Quillen equivalences between models of $(\infty,1)$-categories}

First consider the projection and inclusion functors (on the first component) $p_1:\Delta\times\Delta\to\Delta:([m],[n])\mapsto[m]$ and $i_1:\Delta\to\Delta\times\Delta:[n]\mapsto([n],0)$. They induce an adjoint pair of functors
$$p_1^*:\mathcal S\rightleftarrows\mathcal S^{(2)}:i_1^*$$
between simplicial sets and bisimplicial sets. Under the identification of bisimplicial sets with simplicial spaces, $p_1^*$ sends a simplicial set $X$ into a discrete simplicial space $\tilde X$ with $\tilde X_m$ being the discrete simplicial set generated by $X_m$. $i_1^*$ associates a simplicial space $Y$ with the simplicial set with $n$-simplices given by $(Y_n)_0$.

\begin{thm}
The adjoint pair
$$p_1^*:{\cal QC}\rightleftarrows{\cal CSS}:i_1^*$$
is a Quillen equivalence.
\end{thm}

\begin{proof}
See \cite{jt}.
\end{proof}

Let $\Delta^{|2}=([0]\times\Delta)^{-1}(\Delta\times\Delta)$ where we formally invert all morphisms in $[0]\times\Delta$. There is a canonical map $\pi:\Delta\times\Delta\to\Delta^{|2}$. Since $p_1:\Delta\times\Delta\to\Delta$ sends all morphisms in $[0]\times\Delta$ to invertible morphisms, it factors through $q:\Delta^{|2}\to\Delta$ where $q\pi=p_1$. Define $j=\pi i_1:\Delta\to\Delta^{|2}$. Then $q$ and $j$ restrict $p_1$ and $i_1$ to $\Delta^{|2}$ and induce an adjoint pair of functors
$$q^*:\mathcal S\rightleftarrows{\cal PC}at:j^*.$$
Explicitly, $q^*$ takes a simplicial set $X$ to the discrete bisimplicial space $\tilde X$ with $\tilde X_m$ being the discrete simplicial set generated by $X_m$. $j^*$ takes a precategory $Y$ to the simplicial set $Y_{*0}$.

\begin{thm}
The adjoint pair
$$q^*:{\cal QC}\rightleftarrows{\cal SC}:j^*$$
is a Quillen equivalence.
\end{thm}
\begin{proof}
See \cite{jt}.
\end{proof}

We now show a Quillen equivalence between ${\cal CSS}$ and ${\cal SC}$. We have a natural inclusion of precategories into simplicial spaces $I:{\cal SC}\to{\cal CSS}$. We can construct a right adjoint, the discretization functor $R:{\cal CSS}\to{\cal SC}$ defined by the homotopy pullback square
$$\xymatrix{RW\ar[r]\ar[d]&\mathrm{cosk}(W_{0,0})\ar[d]\\W\ar[r]&\mathrm{cosk(W_0)}},$$
that is we take the discretization of the 0-space of $W$. If $W$ is a complete Segal space, we can explicitly define $RW$ by $RW_0=W_{0,0}$ is a discrete simplicial set, $RW_1$ by the homotopy pullback
$$\xymatrix{RW_1\ar[r]\ar[d]&W_{0,0}\times W_{0,0}\ar[d]\\W_1\ar[r]&W_0\times W_0}$$
and $RW_k=RW_1\times_{RW_0}\cdots\times_{RW_0}RW_1$ for $k\ge2$.

\begin{thm}
The adjoint pair
$$I:{\cal SC}\rightleftarrows{\cal CSS}:R$$
is a Quillen equivalence.
\end{thm}

\begin{proof}
See \cite{bergner3}.
\end{proof}

Note that $I=\pi^*$ as defined above, so we have a commutative triangle $p_1^*=Iq^*$.

With these three Quillen equivalences, we can thus conclude that the categories of quasi-categories, complete Segal spaces and Reedy-fibrant Segal categories are homotopy equivalent.

For the other Quillen equivalences, interested readers can refer to \cite{hs}.

\printbibliography[heading = local]

\end{refsection}
