\chapter{Algebraic structures in higher categories}
\chapterprecistoc{\textup{by} Pietro Vertechi}

\begin{refsection}

\label{weak structures}

\section{Infinity operads}

The main idea behind Moerdijk $\infty$-operads is the following: an $\infty$-operad, whatever that may be, should be characterised by the way in which some elementary operads (namely trees) can be mapped into it. The category of $\infty$-operads should therefore be a particular case of presheaves over $\SOmega$, the category of trees. More precisely:

\begin{definition}
$\SOmega$ is the category whose objects are finite, rooted, non planar trees. Arrows between two trees are morphisms of the corresponding operads. See \cite{Mo-We}
for more details.\\
A dendroidal sets is a functor $\Omega\to \Set$. The category of dendroidal sets will be denoted $\dset$.\\
Given a tree T, we denote by $\Omega[T]$ the corresponding representable dendroidal set.
\end{definition}

\begin{definition}[\cite{Mo-We}]
A monomorphism of dendroidal sets $X\to Y$ is normal if for any tree T, any non degenerate dendrex $y \in Y (T)$ which does not belong to the image of $X(T)$ has a trivial stabilizer $\mathrm{Aut}(T)_y \subseteq \mathrm{Aut}(T)$. A dendroidal set X is normal if the map $\emptyset\to X$ is normal.
\end{definition}

Given an inner edge $e$ in a tree $T$, we get an inclusion
\[
\Lambda^e[T]\to \Omega[T]
\]
Such maps are called inner horn inclusions.

\begin{definition}
A dendroidal set is an $\infty$-operad if it has the right lifting property with respect to inner horn inclusions.
\end{definition}

\begin{remark}
Let $O$ be an operad in the traditional sense (we mean a symmetric multicategory): it can be interpreted as a dendroidal set $N(O):= \Hom_{\mathrm{Operads}}(-,O)$. Such a dendroidal set is an $\infty$-operad. The nerve functor $\mathrm{N} \colon \mathrm{Operads} \to \infty \textrm{-} \mathrm{Operads}$ admits a left adjoint $\Ho$. In analogy with quasi-categories, given an $\infty$-operad $\M$ we call $\Ho(\M)$ its homotopy operad.
\end{remark}

Classically, one may interpret an operad $O$ as an ``algebraic structure''. Given a symmetric monoidal category $\C^{\otimes}$ one can build a corresponding operad $\hat{\C}$in the
following way: the objects of $\hat{\C}$ are the objects of $\C^{\otimes}$ and
\[
\mathrm{Arr}_{\hat{\C}}(A_1,\dots,A_n;B) \coloneqq \mathrm{Arr}_{\C^{\otimes}}(A_1\otimes\dots\otimes A_n;B)
\]
Then O-objects in $\C^{\otimes}$ are simply functors from O to $\hat{\C}$. Furthermore one disposes of a tensor product on operads (called the Boardman-Vogt tensor product)
such that A-objects in B-objects are the same as $A\otimes B$-objects. This is the situation we wish to generalise.

The classical Boardman-Vogt tensor product $\otimes_{BV}$ on operads extends to dendroidal sets in the following way:
\begin{itemize}
\item on trees it is the classical one: $\Omega[S]\otimes\Omega[T] \coloneqq N(S\otimes_{BV}T)$;
\item it is preserves colimits separately in each variable.
\end{itemize}

\begin{proposition}[\cite{Mo-We}]
There is a closed monoidal model structure on dendroidal sets such that:
\begin{itemize}
\item cofibrations are normal monomorphisms;
\item fibrant objects are $\infty$-operads.
\end{itemize}
One should think of weak equivalences between fibrant objects as equivalences of $\infty$-operads.
\end{proposition}

\begin{remark}
There is a Quillen adjunction $i_! \colon \sSet \leftrightarrows \dset \colon i^*$ induced by the forgetful functor $i \colon \SDelta \to \SOmega$. Given an $\infty$-operad $O$ we will refer
to $i^*(O)$ as its underlying $\infty$-category.
\end{remark}

%\begin{definition}
%A symmetric monoidal $\infty$-category is an $\infty$-operad $\C^{\otimes}$ such that, for all n and for all n-uple $A_1,\dots, A_n$ of objects of $\C^{\otimes}$
%the functor $Arr(A_1,\dots,A_n; -): i^*\C^{\otimes}\to sSets$ is representable.
%\end{definition}

\section{Weak algebraic structures}

\begin{definition}
A weak algebraic structure is simply an $\infty$-operad.

Let $\M$ be a weak algebraic structure, let $\C^{\otimes}$ be a $\infty$-operad. Then the category of $\M$-objects in $\C^{\otimes}$ is $i^* \mathrm{Fun}(\mathbb{R}\M,\C^{\otimes})$ where $\mathbb{R}$ is a cofibrant replacement of $\M$. It will be denoted $\M \textrm{-} \C^{\otimes}$
\end{definition}

The notation $\C^{\otimes}$ is appropriate because the definition is particularly interesting when $\C^{\otimes}$ is the $\infty$-operad associated to a ``symmetric monoidal $\infty$-category'' (the inverted commas are due to the fact that there isn't much theory of symmetric monoidal $\infty$-categories in the literature up to now).\\

Indeed one could think that the definition of $\M$ object is completely useless as monoidal categories in nature do not appear as dendroidal sets. There are several possible solutions to this problem:
\begin{itemize}
	\item Given a ``symmetric monoidal $\infty$-category'' the way they arise in nature, for instance as simplicial monoidal model categories, one could simply construct the corresponding simplicial operad and then, using the equivalence between simplicial operads and $\infty$-operads (see \cite{Ci-Mo2}), obtain the corresponding dendroidal set.

	\item Another possibility is simply to translate all the results in another equivalent setting, e.g. simplicial or Segal operads.
	
	\item One could also try to understand what is a symmetric monoidal quasi-category and how to associate a dendroidal set to it.
\end{itemize}

\begin{definition}
\begin{itemize}
	\item $E_1$ is the dendroidal nerve of the 1-operad corresponding to monoids. $E_{\infty}$ is the nerve of the 1-operad corresponding to abelian monoids. It is interesting to remark that $E_{\infty}(T)$ is a point for all tree T.

	\item $\Cat_{\infty}^{\times}$ is the $\infty$-operad corresponding to quasi-categories with the direct product.

	\item A monoidal $\infty$-category is a $E_1$-$\infty$-category. A symmetric monoidal $\infty$-category is a $E_{\infty}$-$\infty$-category. 
\end{itemize}
\end{definition}

\begin{proposition}
Each tree T can be seen as a symmetric monoidal 1-category $\mathrm{tens}(T)$ in the following way:
\begin{itemize}
	\item objects of tens(T) are the free abelian monoid on edges of T;

	\item Arrows are freely generated by operations on T;

	\item The tensor produc is simply the composition in the free abelian monoid.
\end{itemize}
This functor $\mathrm{tens} \colon \SOmega \to \Cat_\infty^{\mathrm{sym. mon.}}$ induces a functor
\[
\mathrm{tens}^*: \Cat_{\infty}^{\mathrm{sym. mon.}} \to \dset
\]
It can be checked that its image is contained in $\infty$-operads.

We will often omit the notation $\mathrm{tens}^*$ and refer to a symmetric monoidal $\infty$-category or to the corresponding $\infty$-operad in the same way.
\end{proposition}

\begin{remark}
As we promised in the introduction, the notion of weak algebraic structure that we get is manifestly $\infty$-categorical, so if there is an equivalence between two objects, than a weak structure on the first induces a weak structure on the second in a compatible way. This will be of great importance in the applications in derived algebraic geometry.
\end{remark}

\section{Looping and delooping}

Before studying $E_1$ or $E_{\infty}$ $\infty$-categories, it is already interesting to consider $E_1$ or $E_{\infty}$ spaces. The first interesting thing to remark is that $E_1$-spaces are the same as quasi-categories with only one 0-simplex, so that it is particularly easy to construct the looping/delooping adjunction.

\begin{definition}
There is a map $i \colon \SDelta\to E_1 \textrm{-} \sSet$ sending $[n]$ to the discrete free monoid on $n$ elements. It induces a functor $E_1 \textrm{-} \sSet \to \sSet$ in the following way:
\[
M \mapsto ([n] \mapsto \Hom(i(n),M)_0 )
\]
We call such functor the delooping functor, denoted $\mathrm{B}$.
\end{definition}

\begin{proposition}
It can be show that if $M$ is a weak monoid, then $\mathrm{B}M$ is an $\infty$-category with one element. We say that a monoid $M$ is \emph{grouplike} if $\mathrm{B}M$ is a Kan complex.
\end{proposition}

\begin{remark}
So far we have a functor $\mathrm{B} \colon E_1 \textrm{-} \sSet \to \Cat_{\infty,\bullet}$. The easiest way to build its adjoint $\Omega$ is to observe that $\Cat_{\infty, \bullet}$ is equivalent to the model category $\mathbf{sCat}_{\bullet}$ of pointed simplicial categories. We have the obvious functor $\Omega: \mathbf{sCat}_{\bullet} \to E_1\textrm{-} \sSet$ sending a
pointed simplicial category $\C$ to the (strict) monoid $\Hom_{\C}(*,*)$.
\end{remark}

The looping/delooping adjunction identifies $E_1 \textrm{-} \sSet$ with $\infty$-categories with one object and furthermore provides a strictification result for weak simplicial monoids. It is also clear that $\Omega \circ B \simeq Id$, whereas $\mathrm{B}(\Omega X) \simeq X$ only if X is connected.

\begin{proposition}
Let $\M$ be a weak algebraic structure. If $G$ is a $E_1 \otimes \M$-simplicial set, then $\mathrm{B} M$ is an $\M$-simplicial set in a natural way.
\end{proposition}

\begin{proof}
By Freyd's adjoint theorem, $\mathrm{B}$ has a left adjoint $\mathrm{L} \colon \sSet \to E_1 \textrm{-} \sSet$: it is the only colimits preserving functor that extends $i$. Therefore $\mathrm{B}$ is a monoidal functor (as both monoidal structures are cartesian) and so sends $\M \textrm{-} E_1 \textrm{-} \sSet$ to $\M \textrm{-} \sSet$.
\end{proof}

\begin{corollary}
Let $E_{\infty}$ be the operad of abelian monoids. As $E_{\infty} \simeq E_{\infty} \otimes E_1$, the functor $\mathrm{B}$ induces a functor $E_{\infty} \textrm{-} \sSet \to E_{\infty} \textrm{-} \sSet$.
\end{corollary}

\printbibliography[heading = local]

\end{refsection}
