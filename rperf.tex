\chapter{Derived moduli stacks}
\chapterprecistoc{\textup{by} Pieter Belmans}
\begin{flushright}
  Pieter Belmans
\end{flushright}

\begin{refsection}

\section{Introduction}

\section{Preliminaries}
Whenever the terminology ``derived stack'' is used we refer to the~$\mathbf{D}^-$\dash stacks from \cite{hagII}.

\section{Properties of \texorpdfstring{$\RPerf$}{RPerf}}
\begin{definition}
  Let~$A_*$ be a simplicial commutative~$k$\dash algebra. Let~$P^\bullet$ be a~$\nerve(A_*)^\bullet$\dash dg~module\checkthis{cohomological indexing?}. We say it is of \emph{Tor amplitude contained in~$[a,b]$} if for every~$\pi_0(A_*)$\dash module~$M\in\Ob(\pi_0(A_*)\mhyphen\Mod)$ we have
  \begin{equation}
    \HH^i\left( P^\bullet\otimes_{\nerve(A_*)^\bullet}^\LLL M \right)=0
  \end{equation}
  if~$i\notin[a,b]$.
\end{definition}

\begin{definition}
  If we take~$\mathcal{C}=k$ in definition \ref{definition:derived-stack-of-pseudoperfect-objects} we obtain the \emph{derived moduli stack of perfect modules}. It will be denoted~$\RPerf$.
\end{definition}

In the notation of \cite{toen-vaquie} we have~$k=\mathbf{1}$ and~$\RPerf$ is denoted by~$\mathcal{M}_{\mathbf{1}}$.

This derived stack serves as a ``base point'' for all other constructions. One first proves the properties for~$\RPerf$ (see theorem \ref{theorem:main-theorem-RPerf}), then one can try to lift these to derived moduli stacks for arbitraty dg~categories, under some appropriate finiteness conditions (see theorem \ref{theorem:main-theorem}).

\begin{theorem}
  \label{theorem:main-theorem-RPerf}
  The derived moduli stack~$\RPerf$ is locally geometric and locally of finite presentation.

  \begin{proof}[Sketch of a proof]
    For the complete proof, see \cite[proposition 3.7]{toen-vaquie}.
    \begin{enumerate}
      \item Let~$a\leq b$ be integers. We will introduce a full derived substack~$\RPerf^{[a,b]}$ of~$\RPerf$.
        
        For~$A_*\in\Ob(\scAlg{k})$ we see that~$\pi_0(\RPerf(A_*))$ consists of the quasi-isomorphism classes of perfect~$\nerve(A_*)^\bullet$\dash dg~modules. Hence we can define~$\RPerf^{[a,b]}(A_*)$ to be the full subsimplicial set of~$\RPerf(A_*)$ of connected components that correspond to perfect~$\nerve(A_*)^\bullet$\dash dg~modules whose Tor amplitude is contained in~$[a,b]$. We can then write
        \begin{equation}
          \RPerf=\bigcup_{a\leq b}\RPerf^{[a,b]}
        \end{equation}
        and we have reduced the proof to showing that~$\RPerf^{[a,b]}$ is locally geometric and locally of finite presentation. We can moreover prove that~$\RPerf^{[a,b]}$ is~$n$\dash geometric, where~$n=b-a+1$.

      \item 
    \end{enumerate}
  \end{proof}
\end{theorem}

\begin{theorem}
  \label{theorem:main-theorem}
  Let~$\mathcal{C}$ be a dg~category of finite type. Then the derived moduli stack~$\mathcal{M}_{\mathcal{C}}$ is locally geometric and locally of finite presentation.
\end{theorem}

\section{Example: derived moduli stack of perfect complexes on a scheme}
Consider a smooth and proper scheme~$X$ over a field~$k$. By~$\Qcoh_X$ we denote its category of quasicoherent sheaves. In general there is no obvious model category structure on~$\Ch(\Qcoh(X))$. But in this case~$X$ is quasicompact and separated, so by \cite{hovey-sheaves} there exists a model category structure on~$\Ch(\Qcoh_X)$ where the cofibrations are the monomorphisms and weak equivalences are quasi-isomorphisms. There is moreover an obvious~$\Ch(k\mhyphen\Mod)$\dash enrichment, so~$\Ch(\Qcoh_X)$ is a~$\Ch(k\mhyphen\Mod)$\dash model category. Using the internal dg~category as seen in definition~\ref{definition:internal-dg-category} we can define
\begin{equation}
  \LLL_\qcoh(X)\coloneqq\Int\left( \Ch(\Qcoh_X) \right).
\end{equation}
By proposition \ref{proposition:internal-category-is-dg-enrichtment} this category serves as a dg~enrichment of the ``standard'' unbounded derived category of quasicoherent sheaves, in the sense that
\begin{equation}
  \mathbf{D}_\qcoh(X)\cong\Ho\left( \LLL_\qcoh(X) \right).
\end{equation}
If we restrict ourselves to the perfect objects in\fixthis{introduce~$\LLL_\perf$} we obtain the equivalence
\begin{equation}
  \mathbf{D}_\perf(X)\cong\Ho\left( \LLL_\perf(X) \right),
\end{equation}
where now the smoothness comes into play.

In order to apply the theory of derived moduli stacks certain finiteness conditions on the categories must be satisfied. Crucial in this is the result from \cite{bondal-vandenbergh}. One could paraphrase the main result as
\begin{quote}
  Quasicompact and quasiseparated\footnote{The intersection of two affine opens is again affine. This is always the case for separated schemes.} schemes are \emph{derived affine}.
\end{quote}
\begin{flushright}
  \cite[corollary 3.1.8]{bondal-vandenbergh}
\end{flushright}
This means that~$\mathbf{D}_\qcoh(X)$ is equivalent to~$\mathbf{D}(A^\bullet)$, where~$A^\bullet$ is a dg~algebra with bounded cohomology. This algebra is moreover obtained by looking at~$\Ext^\bullet(E,E)$ where~$E$ is a compact generator for~$\mathbf{D}_\qcoh(X)$.

\section{Example: derived moduli stack of pseudoperfect complexes of quiver representations}

\printbibliography[heading = local]

\end{refsection}
