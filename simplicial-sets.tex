\chapter{Simplicial sets}
\chapterprecistoc{\textup{by} Mauro Porta}

\begin{refsection}

In this appendix we will quickly overview the homotopy theory of simplicial sets. The exposition given here is by no means complete, and the interested reader is referred to \cite{may} and to \cite[Chapter 1]{gj} for a more comprehensive treatment.

Simplicial sets have been introduced in order to manipulate in a easier way the theory of CW-complexes, in the sense that they provide a combinatorial model for the same homotopy theory. Using the language introduced in the first Expos\'e, we can make precise this statement by saying that there is a Quillen equivalence
\[
|\cdot| \colon \sset \rightleftarrows \Top \colon \mathrm{Sing}
\]
The functors involved in this equivalence are the geometric realization and the singular complex. We assume that the reader has some experience with $\mathrm{Sing}$ coming from the Algebraic Topology; the first important task of this appendix is, on the other side, to describe the geometric realization and to analyse its most important properties.

Later, we will describe the homotopy theory of $\sSet$ using the internal combinatorics: a particular attention is paid to the formalism of anodyne extensions and to its applications. We will conclude describing the notion of minimal simplicial set and a sketch of the proof that $\sSet$ has a model structure.

\section{The category \texorpdfstring{$\SDelta$}{\textbackslash Delta} and (co)simplicial objects}

\begin{notation}
For every $n \in \N$ let $\mathbf n$ be the category associated to the (unique) linearly ordered set with $n$ elements.
\end{notation}

\begin{defin}
Let $\SDelta$ be the full subcategory of $\Cat$ spanned by the categories $\mathbf n$ as $n$ varies in $\N$.
\end{defin}

\begin{rmk}
$\SDelta$ is thus the category whose objects are finite totally ordered sets and whose arrows are (weakly) increasing functions.
\end{rmk}

The combinatorial nature of $\SDelta$ allows to produce a simple factorization system\footnote{Remember that a factorization system in a category $\mathcal C$ is the datum of two subcategories $\mathcal C_0$ and $\mathcal C_1$ such that every arrow $f$ can be written as $f = gh$, where $g$ is in $\mathcal C_0$ and $h$ is in $\mathcal C_1$.} in this category. In order to do so, we introduce two particular classes of arrows: the \emph{coface} maps $d^i \colon \mathbf n \to \mathbf{n+1}$ ($i \in \{0,1,\ldots,n+1\}$) defined by
\[
d^i(j) = \begin{cases} j & \text{if } j < i \\ j + 1 & \text{if } j \ge i \end{cases}
\]
These are the only strictly increasing arrows from $\mathbf n$ to $\mathbf{n+1}$.

Similarly, we can construct $n+1$ arrows $s^i \colon \mathbf{n+1} \to \mathbf n$ by requiring that exactly two successive numbers have the same image:
\[
s^i(j) = \begin{cases} j & \text{if } j \le i \\ j - 1 & \text{if } j > i \end{cases}
\]
We will call the maps $s^i$ the codegeneracy maps.

\begin{thm} \label{thm factorization epi-mono for Delta}
Let $f \colon \mathbf n \to \mathbf m$ be any arrow in $\SDelta$. Then there are uniquely determined maps $d,s \in \mathrm{Ar}(\SDelta)$ such that
\[
f = d \circ s
\]
where
\begin{equation} \label{eq increasing part}
d = d^{i_1} \circ \ldots \circ d^{i_s} \qquad 0 \le i_s \le \ldots \le i_1 \le m
\end{equation}
and
\begin{equation} \label{eq decreasing part}
s = s^{j_1} \circ \ldots \circ s^{j_r} \qquad 0 \le j_1 < \ldots < j_t < n
\end{equation}
\end{thm}

\begin{rmk}
Before starting the proof, let's observe that if $f \colon \mathbf n \to \mathbf m$ is injective (surjective) as map of sets, then it is monic (epi) in $\SDelta$. This is essentially obvious, and has as simple consequence the fact that any iterated composition of the face maps $d^i$ is monic.
\end{rmk}

\begin{proof}
Let $i_s < \ldots < i_1$ be the elements in $\mathbf m$ not in the image of $f$, and let $j_1 < \ldots < j_t$ be the elements in $\mathbf n$ such that $f(j) = f(j+1)$. Write also
\[
p = n -t = m - s
\]
We claim that with this choice of indexes, equations \eqref{eq increasing part} and \eqref{eq decreasing part} gives the desired factorization:
\[
\xymatrix{ \mathbf n \ar[r]^d & \mathbf p \ar[r]^s & \mathbf m }
\]
In fact, fix $k \in \{0,1,\ldots,n\}$. Assume that $j_h \le k < j_{h+1}$. Then necessarily $f(k)$ is the $(k-h)$-th element in $\mathbf m \setminus \{i_1,\ldots,i_s\}$; therefore
\[
f(k) = d(k-h) = d(s(k))
\]
The factorization is unique because if $f$ miss $i_s,\ldots,i_1$, then $f$ must factor through $d$, hence $f = d s'$, and since $d$ is mono, $s' = s$.
\end{proof}

Theorem \ref{thm factorization epi-mono for Delta} gives a set of generators for the arrows of the category $\SDelta$. The next Theorem gives the relations between them:

\begin{thm} \label{thm presentation for Delta}
Let $\mathcal D$ be the graph whose objects are natural numbers and such that for each $n \in \N$ there are exactly $n+2$ arrows $\delta^i_n$ ($i \in \{0\,\ldots,n+1\}$) with source $n$ and target $n+1$, and exactly $n+1$ arrows $\sigma^i_n$ ($i \in \{0,\ldots,n\}$) with source $n$ and target $n-1$. Then $\SDelta$ is obtained as the quotient of the free category $F \mathcal D$ via the relations
\begin{equation} \label{eq cosimplicial identities}
\begin{gathered}
d^j d^i = d^i d^{j-1} \quad \text{if } i < j \\
s^j s^i = s^i s^{j+1} \quad \text{if } i \le j \\
s^j d^i = \begin{cases} d^i s^{j-1} & \text{if } i < j \\
\text{id} & \text{if } i = j \text{ or } i = j + 1 \\
d^{i-1} s^j & \text{if } i > j+1 \end{cases}
\end{gathered}
\end{equation}
\end{thm}

\begin{proof}
It is an exercise to check that the coface and codegeneracy maps satisfy these relations. We have now to check the universal property of the quotient. Let $G \colon F \mathcal D \to \mathcal C$ be a functor such that the maps $G(\delta^i_n)$ and $G(\sigma^i_n)$ satisfy the cosimplicial identities. Then define $\overline{G} \colon \SDelta \to \mathcal C$ using the factorization epi-mono provided in Theorem \ref{thm factorization epi-mono for Delta}; the simplicial identities allows to check functoriality of this definition. The uniqueness is, finally, clear.
\end{proof}

\begin{defin} \label{def (co)simplicial objects}
Let $\mathcal C$ be any category. A cosimplicial object in $\mathcal C$ is a functor $F \colon \SDelta \to \mathcal C$; a simplicial object is $\mathcal C$ is a functor $F \colon \SDelta^\text{op} \to \mathcal C$.
\end{defin}

\begin{cor}
Let $\mathcal C$ be any category. To give a cosimplicial object $A$ in $\mathcal C$, it is necessary and sufficient to give a sequence of objects $\{A^n\}_{n \in \N}$ together with coface operators $\partial^i \colon A^{n-1} \to A^n$ and codegeneracy operators $\sigma^i \colon A^{n+1} \to A^n$ ($i = 0,\ldots,n$) satisfying the cosimplicial identities \eqref{eq cosimplicial identities}.
\end{cor}

\begin{cor}
Let $\mathcal C$ be any category. To give a simplicial object in $\mathcal C$ it is necessary and sufficient to give a sequence of objects $\{A_n\}_{n \in \N}$ together with face operators $\partial_i \colon A^{n+1} \to A^n$ and degeneracy operators $\sigma_i \colon A_{n-1} \to A_n$ ($i = 0, \ldots, n$) satisfying the simplicial identities
\begin{gather*}
\partial_i \partial_j = \partial_{j-1} \partial_i \quad \text{if } i < j \\
\sigma_i \sigma_j = \sigma_{j+1} \sigma_i \quad \text{if } i < j \\
\partial_i \sigma_j = \begin{cases} \sigma_{j-1} \partial_i & \text{if } i < j \\
\mathrm{id} & \text{if } i = j \text{ or } i = j+1 \\ \sigma_j \partial_{i-1} & \text{if } j > j+1 \end{cases}
\end{gather*}
\end{cor}

\begin{proof}
Just observe that these simplicial identities are exactly the dual of the cosimplicial ones.
\end{proof}

\section{Simplicial sets}

\subsection{Definitions and examples}

According to Definition \ref{def (co)simplicial objects}, a simplicial set is just a presheaf over $\SDelta$. We will denote this category by $\sSet$. The reader should know that this abstract definition won't be of any help in understanding the rich theory of simplicial sets. However, it allows to develop quickly a large amount of theory.

\begin{defin}
Let $K \in \sSet$. For each $n \in \N$, the set $K_n := K(\mathbf n)$ is called the set of $n$-simplices in $K$; an element $\alpha \in K_n$ is said to be an $n$-simplex of $K$.
\end{defin}

For example, our experience in classical category theory leads us to consider immediately special simplicial sets, namely those corresponding to representable functors.

\begin{defin}
Let $n \in \N$. The standard $n$-simplex $\Delta^n$ is by definition the representable functor
\[
\Delta^n := \Hom_{\SDelta}(-,\mathbf n)
\]
\end{defin}

\begin{defin} \label{def face}
Let $n \in \N$. For every $0 \le i \le n$, the $i$-th face of $\Delta^n$ is defined to be the subobject represented by the map $d_i := (d^i)^* \colon \Delta^{n-1} \to \Delta^n$.
\end{defin}

\begin{prop} \label{prop simplicial morphisms with Yoneda identification}
Let $K \in \sSet$ be a simplicial set. Then
\[
K_n = \Hom_{\sSet}(\Delta^n,K)
\]
Moreover, if $f \colon \mathbf m \to \mathbf n$ is any map in $\SDelta$ and $x \in K_n$ corresponds to $\omega \colon \Delta^n \to K$, then $K(f)(x)$ corresponds to $\omega \circ f_*$.
\end{prop}

\begin{proof}
It's simply Yoneda lemma.
\end{proof}

\begin{defin}
Let $K \in \sSet$ be a simplicial set and let $\omega$ be a simplex in $K$. We say that $\omega$ is degenerate if it can be written as $K(s)(\omega')$ where $s \colon \mathbf m \to \mathbf n$ is a composition of degeneracy maps. We say that $\omega$ is non-degenerate if it is not degenerate.
\end{defin}

\begin{defin} \label{def simplicial set of finite dimension}
Let $K \in \sSet$ be a simplicial set. We say that $K$ is \emph{finite dimensional} (or of \emph{finite dimension}) if there exists a non-negative integer $N \in \N$ such that every non-degenerate simplex of $K$ is of degree $\le N$.
\end{defin}

\subsection{Sub-simplicial sets, skeleta}

Usually, in presence of some kind of algebraic structure, the notions of generator and sub-structure play an important role. Simplicial sets aren't an exception to this rule.

Let's begin with the notion of sub-simplicial set. If $\mathbf A$ is an algebraic category of some sort (e.g. abelian groups) and we consider the category of presheaves $\mathbf A^{\mathcal C^\mathrm{op}}$, we inherit in a natural way a notion of sub-structure directly from the category of $\mathbf A$: if $\{F_i\}_{i \in I}$ are sub-presheaves of a given presheaf $F$ it makes sense to consider the intersection of this family. It follows that if we assign (generalized) elements
\[
\{m_k \colon A_k \to F(C_k)\}_{k \in J}
\]
it also makes sense to consider the smallest sub-presheaf $G$ of $F$ such that for each $k \in J$ it holds a factorization
\[
m_k \colon A_k \to G(C_k) \to F(C_k)
\]
This $G$ will be called the sub-presheaf $G$ of $F$ generated by the elements $m_k$:

\begin{defin}
Let $\mathcal A$ be a category with pullbacks and let $F \colon \mathcal C^{\mathrm{op}} \to \mathcal A$ be an $\mathcal A$-valued presheaf. Let $\{m_k \colon A_k \to F(C_k)\}_{k \in J}$ be a family of generalized elements of $F$; we define the $\mathcal A$-valued presheaf generated by $m_k$ to be the intersection of all the sub-presheaves of $F$ containing the elements $m_k$.
\end{defin}

For simplicial sets, the Theorem \ref{thm presentation for Delta} allows a simple description of the sub-simplicial set generated by a number of simplexes of a given simplicial set.

\begin{prop} \label{prop sub simplicial set}
Let $X$ be a simplicial set and let $\{\omega_i\}_{i \in I}$ be a family of simplices of $X$. The sub-simplicial set $Y$ generated by those simplices is characterized as follows:
\[
\Hom_{\sSet}(\Delta^n,Y) = \{\omega \in X_n \mid \exists f \colon \mathbf n \to \mathbf m, \: \exists \: \omega_i \in X_m \text{ such that } \omega = X(f)(\omega_i) \}
\]
\end{prop}

\begin{proof}
It's the standard model-theoretic proof: the right-hand side is contained in all the sub-presheaves of $X$ containing the family $\{\omega_i\}_{i \in I}$; since it is a sub-presheaf on its own, it is exactly the sub-presheaf generated by those elements by definition.
\end{proof}

\begin{defin}
The \emph{boundary of the standard $n$-simplex} is by definition the sub-simplicial set $\partial \Delta^n$ of $\Delta^n$ generated by the ($n-1$)-simplexes of $\Delta^n$.
\end{defin}

\begin{defin}
Let $n \in \N$ and let $0 \le i \le n$ be an integer. Define $\Lambda^n_i$ to be the sub-simplicial set of $\Delta^n$ generated by all the ($n-1$)-simplexes of $\Delta^n$ but the $i$-th face (cfr. Definition \ref{def face}).
\end{defin}

\begin{defin}
Let $K$ be a simplicial set and let $N \in \N$ be a non-negative integer. We define the $N$-th skeleton of $K$ to be the sub-simplicial set of $K$ generated by the simplexes of $K$ of degree $\le N$. We will denote the $N$-th skeleton of $K$ by $\mathrm{sk}_N(K)$.
\end{defin}

\begin{rmk}
$\mathrm{sk}_N(K)$ is also the sub-simplicial set generated by $K_N$. In particular we have the identity:
\[
\partial \Delta^n = \mathrm{sk}_{n-1}(\Delta^n)
\]
\end{rmk}

%\subsection{An adjunction}
%
%Denote by
%\[
%c \colon \Set \to \sSet
%\]
%the diagonal functor sending a set $X$ into the constant simplicial set $c(X)$ associated to $X$.
%
%\begin{lemma} \label{lemma set sset adjoints}
%The constant simplicial functor $c \colon \Set \to \sSet$ has as right adjoint the projection functor $p \colon \sSet \to \Set$ sending every simplicial set $K$ into $K_0 = \Hom_{\sSet}(\Delta^0,K)$.
%\end{lemma}
%
%\begin{proof}
%Let $K$ be any simplicial set. Then we have a natural inclusion
%\[
%\varepsilon_K \colon c(K_0) \to K
%\]
%This defines obviously a natural transformation $\varepsilon \colon c \circ p \to \mathrm{Id}_{\sSet}$. If $X$ is any set and we are given $f \colon c(X) \to K$ as in the following diagram:
%\[
%\xymatrix{
%K_0 & c(K_0) \ar[r]^{\varepsilon_K} & K \\ X \ar@{.>}[u]^{\exists ! \:g} & c(X) \ar@{.>}[u]^{c(g)} \ar[ur]_f
%}
%\]
%to complete the proof we only have to show that $g$ exists uniquely as indicated. Necessarily, the commutativity of the diagram on the right forces
%\[
%g = f_0
%\]
%and this works because $c(X)$ is concentrated in degree 0.
%\end{proof}

\section{Geometric realization} \label{geometric realization}

So far we looked at simplicial sets as purely combinatorial objects. In this section, we explain what is the geometric intuition behind them using the geometric realization functor $|\cdot| \colon \sSet \to \Top$. The idea is that this functor uses the combinatorial data encoded by a simplicial set $K$ in order to build a CW-complex $|K|$ whose $n$-cells are in bijection with non-degenerates $n$-simplices of $K$; moreover, this functor has several nice properties which can be often used in order to prove statements about simplicial sets avoiding the hard combinatorics involved.

Remember first of all that in our conventions, $\Top$ denotes the cartesian closed category of compactly generated and Hausdorff topological spaces. Observe then that we have a cosimplicial object
\[
S \colon \SDelta \to \Top
\]
sending $\mathbf n$ to the standard $n$-simplex $\Delta_n$ in $\R^n$. We can define the geometric realization as the left Kan extension
\[
\xymatrix{
\SDelta \ar[r]^{S} \ar[d]_{\mathbf y} & \Top \\
\sSet \ar[ur]_{|\cdot|}
}
\]
Explicitly, this is the unique functor defined by the condition of sending $\mathbf n$ in $\Delta_n$ and commuting with colimits. As usual, in presence of a left Kan extension, we can produce a right adjoint to $|\cdot|$:
\[
\mathrm{Sing} \colon \Top \to \sSet
\]
simply by setting
\[
\mathrm{Sing}(X) := \Top(\Delta^{\bullet}, X)
\]

\begin{lemma}
The adjunction $|\cdot| \dashv \mathrm{Sing}$ holds.
\end{lemma}

\begin{proof}
This is a straightforward exercise.
\end{proof}

The following theorem summarizes the main properties of the geometric realization functor $|\cdot|$:

\begin{thm} \label{thm exactness of geometric realization}
The geometric realization functor $| \cdot | \colon \sSet \to \mathbf{CGHaus}$ commutes with finite limits and with colimits. Moreover it reflects isomorphisms.
\end{thm}

\begin{proof}
The previous lemma implies readily that $| \cdot |$ commutes with colimits. A formal argument can be used to show that it commutes with finite products as well, so that we are left to check that it commutes with equalizers. We refer to \cite[Ch. III.3]{gz} for the proof of this statement, as well as for the proof of the last one.
\end{proof}

It is particularly useful that $| \cdot |$ \emph{reflects} isomorphism. This can be used to show that certain colimits are pushouts in the category $\sSet$, without the need for hard combinatorial arguments. The following lemma explains how to do this:

\begin{lemma} \label{lem colimit reflection}
Let $\mathcal C$ be a cocomplete category and let $F \colon \mathcal C \to \mathcal D$ be a functor preserving colimits. If in addition $F$ reflects isomorphisms, then $F$ reflects colimits.
\end{lemma}

\begin{proof}
Straightforward.
\end{proof}

\begin{cor} \label{cor colimit in sset}
The geometric realization functor $| \cdot | \colon \sSet \to \mathbf{CGHaus}$ reflects colimits.
\end{cor}

\begin{proof}
It is a consequence of Theorem \ref{thm exactness of geometric realization} and Lemma \ref{lem colimit reflection}, since $\sSet$ is cocomplete.
\end{proof}

\begin{exercise}
Show that there exists a coequalizer diagram as the following:
\[
\xymatrix{
\coprod_{i = 0}^{\frac{n(n+1)}{2}} \Delta^{n-2} \ar@<.5ex>[r] \ar@<-.5ex>[r] & \coprod_{i = 0}^{n} \Delta^{n-1} \to \partial \Delta^n
}
\]
\end{exercise}

\begin{cor}
For every simplicial set $K$, its geometric realization $|K|$ is a CW-complex.
\end{cor}

\begin{proof}
See \cite[Proposition I.2.3]{gj}.
\end{proof}

\section{Kan fibrations}

As it was claimed in the introduction, the category $\sSet$ has a homotopical content which is equivalent to the one carried by $\Top$; however, the existence of this equivalence probably wouldn't be so interesting if it weren't possible to describe a huge part of this homotopical information in a combinatorial way.

In this section we describe the first bit of the internal homotopy theory of $\sSet$: we introduce the class of maps which will become the fibrations of a model structure.

\begin{defin} \label{def kan fibration}
A map of simplicial sets $f \colon F \to G$ is said to be a \emph{Kan fibration} if it has the right lifting property with respect to every map $\Lambda^n_i \to \Delta^n$, for every $n \in \N$ and every $0 \le i \le n$.
\end{defin}

This combinatorial definition is often not too hard to check in practice. For example, a basic fact that is used over and over through these notes is the following:

\begin{lemma} \label{lemma group fibration}
Let $f \colon G \to H$ be a morphism of simplicial groups. If $f$ is a surjection, then $f$ is a Kan fibration (forgetting the group structure).
\end{lemma}

\begin{proof}
See \cite[Lemma 8.2.8]{weibel}.
\end{proof}

\section{Anodyne extensions}

The very definition of simplicial sets makes clear that $\sSet$ is a cartesian closed category (because it is a category of presheaves). The internal hom can be described quite explicitly:
\[
\mathbf{hom}(X,Y) := \Hom_{\sSet}(X \times \Delta^{\bullet}, Y)
\]

\begin{exercise}
Verify that the adjunction $- \times X \dashv \mathbf{hom}(X,-)$ holds for every $X$. (\emph{Hint}: every simplicial set can be written as colimit of standard simplices).
\end{exercise}

\begin{notation}
We will denote $\mathbf{hom}(X,Y)$ with the lighter notation $Y^X$.
\end{notation}

A key point in proving that $\sSet$ has a model structure is to show that $\mathbf{hom}$ behaves well with respect to the class of Kan fibrations. The main result of this section will imply in particular that whenever $K$ is a Kan complex and $L$ is any simplicial set, then $K^L := \mathbf{hom}(L,K)$ is again a Kan complex. This result has been known since the very beginning of the theory of simplicial sets, and proofs can be found in \cite{gz} and in \cite{may}; however, these proofs use sophisticated combinatorial arguments based on a careful analysis of the non-degenerate simplices of $X \times Y$ via $(p,q)$-shuffles. Nowadays, the theory of anodyne extensions makes these arguments to appear unnecessarily complicated.

\begin{defin}
A class of monomorphisms $S$ in $\sSet$ is said to be \emph{saturated} if it satisfies the following requirements:
\begin{enumerate}
\item $S$ is closed under isomorphisms;
\item $S$ is closed under retracts;
\item $S$ is closed under pushouts;
\item $S$ is closed under countable compositions and arbitrary disjoint unions.
\end{enumerate}
\end{defin}

\begin{lemma} \label{lemma llp saturation}
\begin{enumerate}
\item Let $f \colon F \to G$ be a map between simplicial sets. The class of monomorphisms with the LLP with respect to $f$ is saturated;

\item the intersection of saturated classes in $\sSet$ is again saturated;

\item if $A$ is a saturated class of monomorphisms, $A = \mathrm{LLP}(\mathrm{RLP}(A))$.
\end{enumerate}
\end{lemma}

\begin{proof}
The first two statement are easy exercises left to the reader. We prove the last one: the inclusion $A \subset \mathrm{LLP}(\mathrm{RLP}(A))$ holds by definition; the other one is a consequence of the small object argument (see Corollary \ref{cor weakly saturation}).
\end{proof}

In particular, if we are given a class $S$ of monomorphisms we can consider the saturated class generated by $S$ as the intersection of all the saturated classes containing $S$ (the reader will verify that there is always at least one saturated class containing $S$).

\begin{thm} \label{thm anodyne}
The following three classes of monomorphisms have the same saturation:
\begin{enumerate}
\item $\mathbf B_1$, the class of all inclusions $\Lambda^n_k \to \Delta^n$ for $n \in \N$, $n > 0$ and $0 \le k \le n$;
\item $\mathbf B_2$, the class of all inclusions
\[
(\partial \Delta^n \times \Delta^1) \cup (\Delta^n \times \{e\}) \to \Delta^n \times \Delta^1
\]
for $e \in \{0,1\}$;
\item $\mathbf B_3$, the class of all inclusions
\[
(Y \times \Delta^1) \cup (X \times \{e\}) \to X \times \Delta^1
\]
induced by inclusions $Y \to X$.
\end{enumerate}
\end{thm}

\begin{proof}
See \cite[Proposition I.4.2]{gj}.
\end{proof}

\begin{cor} \label{cor anodyne 1}
A map is a fibration if and only if it has the RLP with respect to all anodyne extensions. Similarly, a monomorphism is anodyne if and only if it has the LLP with respect to every fibration.
\end{cor}

\begin{proof}
If $f$ has the RLP with respect to all anodyne extensions, it has in particular the RLP with respect to all the horn inclusions $\Lambda^n_k \subset \Delta^n$, $n \in \N$, $n > 0$ and $0 \le k \le n$; hence $f$ is a fibration. Conversely, if $f$ is a fibration, then the class of monomorphisms $S_f$ having the LLP with respect to $f$ is a saturated class thanks to Lemma \ref{lemma llp saturation}.1; since it contains $\mathbf B_1$, it follows that $S_f$ contains all the anodyne extensions, hence the thesis.

For the second statement, denote by $A$ the class of anodyne extensions. Then we have that $A$ is saturated by definition and moreover if we denote by $F$ the class of Kan fibrations, we have
\[
F := \mathrm{RLP}(A)
\]
so that
\[
A = \mathrm{LLP}(\mathrm{RLP}(A)) = \mathrm{LLP}(F)
\]
because of Lemma \ref{lemma llp saturation}.3.
\end{proof}

\begin{cor} \label{cor anodyne 2}
If $i \colon K \to L$ is an anodyne extension and $j \colon X \to Y$ is an arbitrary inclusion then the induced map
\[
i * j \colon (K \times Y) \coprod_{K \times X} (L \times X) \to L \times Y
\]
is anodyne.
\end{cor}

\begin{proof}
We fix $j$ and consider the class $T_j$ of monomorphisms $f \colon K' \to L'$ such that $f * j$ is anodyne. We claim that $T_j$ is saturated; indeed, adjoint nonsense shows that a monomorphism $f \colon K' \to L'$ is in $T_j$ if and only if it has the LLP with respect to
\[
F^L \to F^K \times_{G^K} G^L
\]
for every Kan fibration $F \to G$. Lemma \ref{lemma llp saturation}.1 and .2 imply then that $T_j$ is saturated. It can be checked that it contains the class $\mathbf B_2$ (see \cite[Corollary I.4.6]{gj} for this), hence we conclude from Theorem \ref{thm anodyne} that $T_j$ contains all the anodyne extensions.
\end{proof}

\begin{cor} \label{cor anodyne 3}
If $i \colon K \to L$ is an inclusion and $p \colon X \to Y$ is a Kan fibration, then the natural map
\[
i \star p \colon X^L \to X^K \times_{Y^K} Y^L
\]
is again a Kan fibration.
\end{cor}

\begin{proof}
Corollary \ref{cor anodyne 1} implies that the map $i \star p$ is a Kan fibration if and only if it has the RLP with respect to every anodyne extension $j \colon A \to B$. Adjoint nonsense shows that this is equivalent to say that $p$ has the RLP with respect to every map
\[
i * j \colon (K \times B) \coprod_{K \times A} (L \times A) \to L \times B
\]
Since $A \to B$ is anodyne, Corollary \ref{cor anodyne 2} implies that $i * j$ is anodyne, so that Corollary \ref{cor anodyne 1} again shows that $p \in \mathrm{RLP}(i * j)$, completing the proof.
\end{proof}

\printbibliography[heading = local]

\end{refsection}
