\chapter{Higher derived stacks}

\begin{refsection}

\begin{flushright}
Pietro Vertechi
\end{flushright}

\section{Higher stacks}

The classical definition of sheaf can be generalised in two different directions: first of all the arrival category of sets may be replaced by the 
$\infty$-category of $\infty$-groupoids (simplicial sets with the Kan-Quillen model structure). Furthermore instead of a site in the classical sense we can 
take a more general $\infty$-category with a topology.\\
\subsection{Definition and comparison with the classical notion}
\begin{defin}
A topology on an $\infty$-category $\mathcal C$ is simply a topology on its homotopy category $\Ho(\mathcal C)$
\end{defin}

\begin{rmk}
This is the natural definition, because it would be absurd if the topology would distinguish between homotopically equivalent sieves: if one is covering, so must be the other.
\end{rmk}


The $\infty$-category of prestacks on $(\mathcal C,\tau)$ is simply the category of functors from $\mathcal C^{op}$ to $sSets$, which we denote $\hat{\mathcal C}$.
That $\infty$-category correspond to a model category, whose weak equivalences and fibrations are defined objectwise.\\

\begin{rmk} $\hat{\mathcal C}$ possesses limits and colimits. Furthermore
$\hat{\mathcal C}$ is characterised by a universal property: given an $\infty$-category $\mathcal D$ with colimits, $Fun_{cont}(\hat{\mathcal C}, \mathcal D)\simeq Fun(\mathcal C,\mathcal D)$, where
$Fun_{cont}$ are colimit preserving functors.\\
It's interesting to remark that this is the same universal property that defines the category of presheaves in the 1-categorical world.
\end{rmk}

In classical sheaf theory it is possible to define the category of sheaves without actually knowing what a sheaf is, simply by localising the category of
presheaves with respect to local equivalences. We will use the same technique in this setting, so the first step is understanding how to generalise the
notion of local equivalence.\\
\begin{defin}
Let $F:\mathcal C\to\mathcal D$ be a map between two $\infty$-categories. We say that F is essentially k-surjective if for all parallel (k-1)-arrows $\phi,\psi\in \mathcal C$ and k-arrow $\chi \in \mathcal D$
between $F(\phi)$ and $F(\psi)$ there exist a k-arrow $\eta\in\mathcal C$ from $\phi$ to $\psi$ such that $F(\eta)$ and $\chi$ are homotopically equivalent.
\end{defin}
A map between two $\infty$-categories is an equivalence if and only if it is essentially k-surjective for all k. For sets it is equivalent to say that a function
between sets is an equivalent if and only if it is 0-essentially surjective (that is to say surjective) and 1-essentially surjective (that is to say injective).\\
It is particularly easy to explain what it means to be locally essentially k-surjective, so we will say that a morphism between prestacks is a local equivalence
if it is locally essentially k-surjective for all k.

\begin{defin}
Let $f: F\to G$ be a morphism of prestacks.\\
1) f is locally essentially k-surjective if it has the following property:\\
for any object $X\in Ob(\mathcal C)$ and any morphism $h_x\to G$ there exist a covering sieve R of X such that for any morphism $h_u\to h_x$ in R there is an arrow
$h_u\to F$ making the following diagram commute up to homotopy:

\begin{displaymath}
\xymatrix{
F^{\Delta_k} \ar[r] &  F^{\partial\Delta_k}\times_{G^{\partial\Delta_k}}G^{\Delta_k}\\
 h_u \ar[u] \ar[r] & h_x\ar[u]
}
\end{displaymath}
we are simply saying that any object of G lifts locally and up to homotopy to an object of F.\\
2) f is a local equivalence if it is locally essentially k-surjective for all k.
\end{defin}

\begin{rmk}
If F and G are 0-presheaves, we are just saying that a map between presheaves is a local equivalence iff:
\begin{itemize}
 \item for all X and $x\in G(X)$, there is a covering of X in which the restrictions of X come from elements of F
\item for all X, if we take $x,y\in F(X)$ which are sent on the same element of G(X) than there is a covering of X in which they are equal.
\end{itemize}

\end{rmk}

\begin{defin}
A stack is a prestack which is local with respect to local equivalences.\\
The $\infty$-category of stacks is the full subcategory of $\hat{\mathcal C}$ spanned by stacks. 
\end{defin}

\begin{prop}[{\cite{tv}}]
The forgetful functor from $St(\mathcal C)$ to $\hat{\mathcal C}$ has an exact left adjoint, called the sheafification functor, which exhibits $St(\mathcal C)$ as the localisation of $\hat{\mathcal C}$
with respect to local equivalences. 
\end{prop}

\begin{proof} [sketch]
As far as I know it is quite difficult to prove this statement directly in the setting of higher categories. One should rather find a model category which correspond to
$\hat{\mathcal C}$, for example the category of functors of simplicial categories between a simplicial category equivalent to $\mathcal C$ and the simplicial category of simplicial sets.
The model structure is the projective model structure (fibrations and weak equivalences are defined objectwise). Then, in this model category, one disposes of the general
machinery of Bousfield localisation which produces a Quillen adjunction between the model category of prestacks and the model category of stacks. When passing to $\infty$
categories the Quillen adjunction induces an adjunction of $\infty$
categories.
\end{proof}

\begin{prop}[{\cite{hagII}}]
We might have started, rather than with weak equivalences, with another (more restrained) class of functions, namely maps of the tipe $|U_*|\to h_X$ where $U_*$ is a split
hypercovering of X. In HAG I it is proved that this set of maps and local equivalences induce the same left Bousfield localisation, so one has a more explicit description
of stacks: they are prestacks such that, for every object X and split hypercovering $U_*\to X$ the natural morphism
$$F(X)\to\lim_{[n]\in\Delta} F(U_n)$$
is an equivalence.
\end{prop}

\begin{rmk}
Thanks to the previous proposition, we have a nice characterisation of colimit-preserving functors from $St(\mathcal C)$:\\
if $\mathcal{D}$ is any $\infty$-category with colimits, a functor $\mathcal C\to\mathcal{D}$ can be extended to a unique colimit preserving functor $\hat{\mathcal C}\to\mathcal{D}$ which
factorises through a (always colimit preserving) functor $St(\mathcal C)\to \mathcal{D}$ iff for all $X\in\mathcal C$ and for all hypercovering $U_*\to X$ the natural map
$$\mathrm{col}_{[n]\in\Delta^{op}} F(U_n)\to F(X)$$
is an isomorphism.\\
Furthermore the adjunction $St(\mathcal C)\rightleftarrows \hat{\mathcal C}$ assures the existence of limits and colimits in $St(\mathcal C)$, so we can characterise such category by a
universal property.
\end{rmk}

\begin{rmk}
Classical sheaves and stacks are a particulare case of this formalism. In general we define an n-stack to be a stack which takes values in the category of n-truncated
simplicial sets (that is to say simplicial sets whose homotopy is concentrated in degrees $\le n$). The inclusion of n-truncated simplicial sets in simplicial sets has a left 
adjoint, denoted $t_{\le n}$ which induces an adjunction $n-St\rightleftarrows St$. It can be verified that 0-stacks are sheaves and 1-stacks are stacks in the classical sense.
\end{rmk}

One might ask why we said we were inverting morphisms that are locally essentially k-surjective for all k and not morphisms that are locally equivalences, which would have been easier to define.
The fact is, a morphism which is locally essentially k-surjective for all k is not locally an equivalence, because a priori we can't find a single covering on which it
is k-surjective for all k at the same time. Of course for truncated stacks there is no difference.\\
The problem with the second choice (inverting local equivalences) is that it produces a topos which is not hypercomplete, that is to say a map $X\to Y$ might induce
isomorphisms on the $\pi_0$ and on all homotopy groups for all base points and still not be an equivalence. Instead, with the definition we've chosen, we have:

\begin{prop}
$St(\mathcal C)$ is hypercomplete.
\end{prop}


\subsection{Sheaves of model categories}

Actually, from these definition it may be rather difficult to prove that a given functor $S^{op}\to sSets$ is a stack. There is, however, a practical way of doing so
 in the case in which S is the $\infty$-category associated to a cofibrantly generated model category T.
\begin{defin}
A left Quillen presheaf is a functor from T to LModCat, the category of model categories in which arrows are left Quillen functors.\\
More explicitely to each object $X\in T$ we associate a model category M(X) and to each arrow $f:X\to Y$ we associate a left Quillen functor $f^*:M(Y)\to M(X)$.\\
Furthermore we demand $(f\circ g)*=f^*\circ g^*$ for any pair of composable arrows.
\end{defin}

Given a split hypercovering $U_*\to X$ we can consider the cosimplicial diagram $[n]\mapsto M(U_n)$ ($U_n$ is a coproduct of representables, and we define $M(U_n)$ to be
 the product of the values of M on those representables).\\
Let's define the category of weak descent data, denoted $M^U$. Its objects are families of objects $x_n\in M(U_n)$ with morphisms $\phi_u:u^*(x_m)\to x_n$ for each 
map $u:U_n\to U_m$ induced by a map $[m]\to [n]$
satisfying the usual cocycle condition. Its morphisms are families of morphisms $f_n: x_n\to y_n$ such that $f_i\circ \phi_u=\phi_u\circ f_j$ for any map $[i]\to[j]$.\\
There is a model category structure on $M^U$ in which fibrations and weak equivalences are defined levelwise.\\
An object $x\in M^U$ is called homotopy cartesian if the maps $\phi_u$ induce isomorphims in the homotopy category $\Ho(M_n)$. We denote $M^U_{cart}$ the full subcategory
spanned by homotopy cartesian objects.

\begin{defin}
M is a sheaf of model categories if, for any $X\in T$ and any hypercovering $U_*\to X$ the natural map $M(X)\to M^U$ induces an equivalence of categories $\Ho(M(X))\to \Ho(M^U_{cart})$
\end{defin}

\begin{thm}[{\cite{hagII}}]
\label{quillen sheaf}
If M is a sheaf of model categories, then $X\to |W(X)^c|$ is a stack. 
\end{thm}


\section{Derived Algebraic Geometry}

Classically, it is sometimes useful to substitute a k-scheme with the functor $k-Alg\to Sets$ it represents, the so called functor of points.\\
The name comes from the fact that if a scheme is given by a system of equations with coefficients in k, than the functor associates to each k-algebra A the set
of solutions of those equations in A. This is a rather useful point of view even in the classical setting: for example a group structure on a scheme $G$
is simply the datum of a group structure on each $G(A)$ in a compatible way and if we are given another scheme X, an action of G on X is simply an
action of $G(A)$ on $X(A)$ for all A compatibly.\\
Unfortunately in this setting quotienting by a group action is a rather poorly behaved operation.\\
A solution to this problem is to enlarge the setting and consider not only sheaves but stacks as well. Besides, to improve intersections as well (classically the schematic
intersection loses some information), we will consider the $\infty$-category of DG k-algebras rather than the category of k-algebras.\\

\subsection{Basic definitions}

\begin{defin}
Let $Mod_k$ denote the monoidal category of unbounded complexes of k-modules with the projective model structure.
\end{defin}

\begin{prop}
$Mod_k$ is a simplicial model category in the following way: the space $\Map(A,B)$ is the simplicial k-module associated to the truncation of the function complex between
A and B. By abuse of notation we will call $Mod_k$ also the $\infty$-category spanned by fibrant cofibrant objects in $Mod_k$
\end{prop}

\begin{defin}
$DG_kAlg_{\le 0}$ is the $\infty$-category of coconnective differential graded algebra.
\end{defin}

In order to do some algebraic geometry on this category, we need to generalise the standard definitions of algebraic geometry to this new context (following To\"en).\\
\begin{defin}
A morphism $f:A\to B$ is flat (resp. smooth, \'etale, a Zariski open immersion) if:\\
1) The induced morphism of affine schemes $\spec H_0(B)\to \spec H_0(A)$ is flat (resp. smooth, \'etale, a Zariski open immersion)\\
2) For any $i>0$ the natural map:
$$H_{-i}(A)\otimes_{H_0(A)}H_0(B)\to H_{-i}(B) $$
is an isomorphism
\end{defin}


\begin{defin}
A finite family of morphisms $\{ f_i:A\to B_i\}$ in $\dalg$ is a ffqc covering if each $f_i$ is flat and the induced morphism of affine schemes
$$\coprod \spec H_0(B_i)\to \spec H_0(A)$$
is surjective
\end{defin}

Now it is possible to define a topology on the $\infty$-category $\dalg^{op}$ saying that a sieve is covering if it contains a ffqc covering.\\
Using the formalism of the previous setting, we define a derived stack to be a stack for this $\infty$ site.\\


\begin{rmk}
There is an obvious adjunction $t_0:\dalg\rightleftarrows k-Alg:i$ which induces an adjunction $i^*:St_k\rightleftarrows Sh(k-Alg):t_0^*$.\\
We say that a stack X is a derived enhancement of a sheaf of k-algebras F if $i(X)=F$. Every sheaf possess a trivial enhancement: $t_0(F)$, which
is always a 0-stack.
However in many cases the best enhancement is not $t_0(F)$, but a more natural one.
\end{rmk}


\subsection{Two examples of stacks}

\begin{defin}[Stack of quasi-coherent sheaves]
Assigning to each DG-algebra A the model category of DG A-modules and to each $\pi: S\to R$ the Quillen functor $S-mod\to R-mod$ corresponding to the extension of scalars
is a sheaf of model categories, denoted QC. So if we consider $X\mapsto |W(QC(X))^c|$ we obtain an example of stack, the classifier of quasi-coherent sheaves. 
\end{defin}

\begin{defin}
We can proceed as above, but restraining our attention to perfect A-modules, that is to say with some finitude conditions: namely a DG-module M is perfect if it is a compact object
in the homotopy category $\Ho(DG-mod)$. We thus get the stack $Perf$, which classifies perfect quasi coherent sheaves.
\end{defin}

\section{Affinisation}

We want to introduce the notion of affine stack (To\"en), which is a homotopical generalisation of the notion of affine scheme and will be useful for dealing
with CW complexes considered as derived stacks.\\
\begin{defin}
The category of affine stacks is the opposite of the category of DG algebras, without any coconnectivity assumptions: $Aff_k:=DGAlg_k^{op}$
\end{defin}

\begin{rmk}
There is an evident map $\spec: Aff_k \to St_k $ given by the Yoneda functor: $\spec(R)(A):=\Hom_{DGAlg_k}(R,A)$
\end{rmk}

As in the classical case, this functor has a left adjoint $\mathcal O: St_k \to Aff_k$ which is the only functor sending each derived affine scheme to the corresponding coconnective
DG algebra and preserving colimits.\\
The monad $\mathrm{Aff}:= \spec\circ\mathcal O$ is called the affinisation functor. Each stack X comes equipped with a universal arrow $X\to\mathrm{Aff}(X)$ through which every arrow
from X to an affine stack factorises.\\

Now we will try to understand what the affinisation functor does on CW complexes. In order to do that, we need to be able to understand how to interprete
CW complexes as derived stacks.\\
\begin{prop}
There is an adjunction $sSets \rightleftarrows DPrSt_k$ where the left adjoint sends each simplicial set 
to the corresponding constant prestack and the right adjoint sends each prestack X to the simplicial set X(k)
\end{prop}

\begin{proof}
$\Map(S,X(k))=\Map(S,\Hom(\spec(k), X))=\Hom(\spec(k)\otimes S, X)$ and $\spec(k)\otimes S$ is the constant simplicial prestack associated to S, as colimits are taken objectwise
and the prestack $\spec(k)$ associates to each DG algebra a point.
\end{proof}

We can compose $sSets \rightleftarrows DPrSt_k$ with the adjunction: $DPrSt_k\rightleftarrows DSt_k$ between the forget and the sheafification functor, as in section 1.\\
We get maps $sSets \rightleftarrows DSt_k$ which allows us to consider simplicial sets as a particular case of derived stacks.
In particular, as left adjoints preserve colimits, and finite simplicial
sets can be expressed as finite colimits of copies of the point, we can recover finite simplicial sets (or equivalently finite CW complexes) as those stacks 
which are finite colimits of copies
of $\spec(k)$.\\
For example the circle $S^1$ is the derived stack:
$$\spec(k)\coprod_{\spec(k)\coprod \spec(k)}\spec(k)$$
Instead general simplicial sets correspond to arbitrary colimits of copies of $\spec(k)$.\\


\section{Cohomology}

In classical algebraic topology cohomology is represented by the Eilenberg-MacLane spaces $K(G,n)=B^nG$. We will be interested in cohomology with coefficient
in k, that is to say $G=\mathbb G_a$ is the additive group. What does $B^n\mathbb G_a$ look like? 

\begin{prop}
For $n\ge 1$ we have $B^n\mathbb G_a\simeq \spec(S_n)$ where $S_n$ is the free
DG algebra on the module $k[-n]$.
\end{prop}
\begin{proof}
Let us consider the functor $\mathcal O$ as a contravariant functor from $St_k$ to $DGMod_k$. It sends pushout squares to pullback squares (which are also pushout squares
as the category $DGMod_k$ is stable).\\
If we consider the pushout diagram:


\begin{displaymath}
\xymatrix{
S^{d-1} \ar[r] \ar[d] &  pt \ar[d]\\
 pt \ar[r] & S^d\\
}
\end{displaymath}

It maps to the pushout/pullback diagram:

\begin{displaymath}
\xymatrix{
\mathcal O(S^{d}) \ar[r] \ar[d] &  k \ar[d]\\
 k \ar[r] & \mathcal O(S^{d-1})\\
}
\end{displaymath}

which allows us to calculate $\mathcal O(S^d)$ (as a DG module) recursively: $\mathcal O(S^d)=k\oplus k[-d]$ \\
It follows from this calculation that
\[
[S^d,\spec S_n]\simeq \pi_0(\Hom_{DGAlg}[S_n,\mathcal O(S^d)])\simeq \pi_0(\Hom_{DGMod}(k[-n],k\oplus k[d])
\]
which is k if d=n and 0 otherwise.\\
On the other hand, $[S^d,B^n\mathbb G_a]\simeq H^n(S^d)$ which is k if d=n and 0 otherwise, so $\spec(S_n)$ and $B^n\mathbb G_a$ have the same homotopy groups 
and only one of them is nontrivial, so they are equivalent.\\

\end{proof}


\begin{thm}
For X a CW complex, $\mathcal O(X)=C^*(X)$ where the right member of the equality is the cochain complex.
\end{thm}

\begin{proof}
$H^n(X)=\Map(X,B^n\mathbb G_a)=\Hom_{DGAlg}(S_n,\mathcal O(X))=H^n(\mathcal O(X))$. We have a natural map $C^*(X)\simeq C^*(\mathrm{col}\spec(k))\to \lim C^*(\spec(k))\simeq \lim k\simeq \mathcal O(X)$\\
\end{proof}
\begin{rmk}
Actually we have a particularly nice description of the algebra $S_1$: it is generated by one element of degree one $\epsilon$ and the algebra structure
is trivial as $\epsilon\epsilon=-\epsilon\epsilon$ (by graded-commutativity) so both sides are equal to 0.\\
So $S_1\simeq k[\epsilon]/(\epsilon^2)\simeq \mathcal O(S^1)$ so $B\mathbb G_a\simeq \mathrm{Aff}(S^1)$
\end{rmk}


\section{Groups and group action}

\begin{defin}
A group stack G is simply a functor $G:\dalg\to A_{\infty}-alg$ such that the induced functor $\dalg\to A_{\infty}-alg\to sSets$ is a stack.\\
Similarly, if G is a group stack and X a stack, a G action on X is an objectwise action af G(A) on X(A) for all $A\in\dalg$ in a compatible way.
\end{defin}

\begin{rmk}
One disposes of the following adjunctions:
$$sSets\rightleftarrows E_1-alg\rightleftarrows sSets_{\bullet}$$
where the first is the free/forget adjunction and the second is the B/$\Omega$ adjunction. Clearly $B\circ free=\Sigma$. So $\Omega\Sigma=\Omega\circ B\circ free=free$, so
if X is a stack the free group on X is simply $\Omega\Sigma X$.
\end{rmk}

\begin{eg}
$X=S^1$. $\Omega S^2 =\Omega\Sigma S^1$ is the corresponding free group, so a $\Omega S^2$ action on X is simply a morphism of stacks $S^1\times X\to X$\\
It is entertaining to understand what the universal group morphism $\Omega S^2\to S^1$ correspond to in more traditional terms. We start from the Hopf fibration $S^1\to S^3\to S^2$
which is classified by the following pullback square:

\begin{displaymath}
\xymatrix{
S^3 \ar[r] \ar[d] &  S^2 \ar[d]\\
 pt \ar[r] & BS^1\\
}
\end{displaymath}

As the functor $\Omega$ is a right adjoint, it preserves pullback squares, so we have:

\begin{displaymath}
\xymatrix{
\Omega S^3 \ar[r] \ar[d] &  \Omega S^2 \ar[d]\\
 pt \ar[r] & S^1\\
}
\end{displaymath}

where the vertical map on the right is the universal group morphism $\Omega S^2\to S^1$ induced by the group structure on $S^1$
\end{eg}

The quotient of a stack by a group stack action is the component-wise homotopical quotient.

\begin{prop}
QC(X/G) is the category of G-equivariant quasi-coherent sheaves over X.
\end{prop}
 
\begin{proof}
We only need to prove it for an affine $X=\spec(A)$. In that case X/G is the colimit of a BG(A) shaped diagram where the only object is $\spec(A)$ and to each arrow we assign
the corresponging element of G. As QC sends colimits to limits, the thesis follows.
\end{proof}

In particular, the category of modules with G-action is simply $QC(*/G)=QC(BG)$

\begin{rmk}
Actually, the preceding reasoning is based on the following rather intresting fact, which is quite interesting per se:\\
if G is a simplicial group, we can form an $\infty$-category that has only an object * and $\Hom(*,*)=G$. This category is an $\infty$-groupoid and so, by the homotopy
hypothesis, it corresponds to a topological space (via the $\Pi_{\infty}$ construction). Such a topological space is the principal G-bundles classifier BG.\\
So, a good theory of $\infty$-categories must include a way of delooping groups. For example, Segal categories (which I personally believe are particularly well suited to
doing derived algebraic geometry) correspond to Segal delooping machine: taking the diagonal of the bisimplicial set corresponding to the Segal category we have constructed
just above.
\end{rmk}

\nocite{hypercover, htt, lurieHA, to1, to2, to3}
\printbibliography[heading = local]

\end{refsection}
